
% Headertext
\headertext{sittning}

\song{Portos visa}        
\melo{Melodi: You can't get a man with a gun}
\author{Text: Tord Andrén}

\songtext{Jag vill börja gasqua
var fan är min flaska?
Vem i helvete stal min butelj?
Skall törsten mig tvinga
en TT börja svinga?
Nej, för fan, bara blunda och svälj.

Vilken smörja!
Får jag spörja:
Vem, för fan, tror att jag är en älg?

Till England vi rider,
och sedan vad det lider
träffar vi välan på någon pub.
Och där skall vi festa,
blott dricka av det bästa
utav whisky och portvin.
Jag tänker gå hårt in
för att pröva på rubb och stubb.}

\songinfo{Från Bergsspexet ”De fyra musketörerna” 1959. Bestämda bergsmän hävdar att visan ursprungligen hette ”Athos visa”.}

\newpage 

%---------------------------------------------------------

\song{Moder Kista}       
\melo{Melodi: Längtan till landet (Otto Lindblad)}
\author{Text: David Larsson}

\songtext{Ack för Kista brinner våra hjärtan
Centrum för I-Teknologi.
Alla sorger sopas under mattan
När vårt Kista vi flanerar i.

Sjung för alla stolta basstationer
Sjung för Forums vackra arkitektur.
Låt oss därför celebrera vår moder;
Kista, du förstärker vår bravur!}

\songinfo{Forums arkitekturs skönhet är extremt subjektiv.}
%---------------------------------------------------------

\song{Gums visa till nubben} 
\author{Text: Torsten Hummel-Gumælius}

\songtext{\leftrepeat Skål kamrater, ty livet är glatt
och snart förgäta vi sorgen.
Vi söpo igår, vi supa idag
och vi tar en sjujäkel i morgon. \rightrepeat
Skål, skål, skål, skååål.}

\newpage 

%---------------------------------------------------------

\song{Jag har aldrig var't på snusen}        
\melo{Melodi: Åh, hur saligt att få vandra}
\tags{Jag har aldrig var't på snusen}

\songtext{
Jag har aldrig var't på snusen
aldrig rökat en cigarr, halleluja!
Mina dygder äro tusen,
inga syndiga laster jag har.
Jag har aldrig sett nå't naket,
inte ens ett litet nyfött barn.
Mina blickar går mot taket,
därmed undgår jag frestarens garn.

\leftrepeat Halleluja, halleluja. \rightrepeat

Bacchus spelar på gitarren,
Satan spelar på sitt handklaver.
Alla djävlar dansar tango,
säg, vad kan man väl önska sig mer?
Jo, att alla bäckar vore brännvin,
Riddarfjärden full av bayerskt öl,
konjak i varenda rännsten
och punsch i varendaste pöl.

\leftrepeat Och mera öl, och mera öl. \rightrepeat}

\newpage

%---------------------------------------------------------

\song{Handelsvisan}       
\melo{Melodi: Åh, hur saligt att få vandra}
\tags{Jag har aldrig var't på snusen}

\songtext{Jag vill aldrig gå på Handels
aldrig tenta företagsekonomi.
Deras IQ den e' Mandels
och förståndet, det har gjort sorti.
De har jätteusla snören
till sitt jättefula draperi.
De kan bara räkna ören
hela Handels e' ett djävla aperi!

\leftrepeat Handels är skit, jag vill ej dit. \rightrepeat

Mammons pojkar är dom alla,
pappas flickor är dom likaså,
går och tror att dom är balla,
fastän dom inget alls kan förstå.
Hela Handels borde rivas,
detta tycker hela vårat lag.
Då så skulle <b><i>[Namn]</b></i> trivas
uppå denna Handels ljuva domedag!

\leftrepeat Åh, vilket drag, på denna dag. \rightrepeat }

\songinfo{Skriven av Team kangaroo till Gerhards-gasque, Fysik KTH, 1977. [Namn] ändras beroende på vilka som sjunger t.ex: Disk sjunger Razor , Fysik sjunger Osquarulda}

\newpage
%-------------------------------------------------------------
\song{Fysikvisan}       
\melo{Melodi: Åh, hur saligt att få vandra}
\tags{Jag har aldrig var't på snusen}

\songtext{Jag vill inte gå på Fysik
aldrig tenta termometerdynamik.
Jag vill inte höra synthmusik,
inte festa som ett tråkigt mattegeek. %note: är det inte, inte festa som ett jävla mattegeek
De ser ut som Televerket
i sin jättefula overall.
De kan bara räkna kvarkar
och nu så hyllar vi Data med en skål!

\leftrepeat Fysik är torrt, jag vill ju bort. \rightrepeat

Einsteins gossar är de alla,
Handels flickor kan de aldrig få.
Går och tror att de har ballar
men det får bli för egen hand om det ska gå.
Nu ska hela Sing-Sing rivas
arkitekt är med på Datas lag.
Televerket ska fördrivas
uppå Konsulatets ljuva domedag

\leftrepeat Å nubbedrag, på denna dag. \rightrepeat}

\songinfo{Gemensamt förkör inför en marskalkssittning på Lärarhögskolan hösten 2000 ledde till att Datasektionen skrev spexet som Fysiksektionen sedan framförde. Om den inte sjungs på nubbedraget sjunger man "Å nubbedrag, nån gång i mars".} \\

\newpage
%---------------------------------------------------------
\song{Jag vill aldrig gå i Kista}        
\melo{Melodi: Åh, hur saligt att få vandra}
\tags{Jag har aldrig var't på snusen}

\songtext{
Jag vill aldrig gå i Kista,
aldrig tenta programmeringsmetodik.
vi kan bara iterera
bara festa som ett jävla dataspel (World of Warcraft)!

Programmering det är Python, 
Forums fuskbygge faller i tu.
Foo Bar lockar dock var onsdag 
därför hänger jag kvar här ännu.

\leftrepeat Åh Kista drägg, dom har ju skägg \rightrepeat}

\songinfo{* Externt itererar Disk KM ”vi kan bara iterera” i oändligheten, eller tills att någon ropar ut "break;"! Internt sjunger DISK KM hela.} \\\

%---------------------------------------------------------

\song{Härjarevisan}       
\melo{Melodi: Gärdebylåten}
\author{Text: Hans Alfredson}

\songtext{Liksom våra fäder, vikingarna i Norden
drar vi landet runt och super oss under borden.
Brännvinet har blitt ett elixir för kropp såväl som själ.
Känner du dig liten och ynklig på jorden,
växer du med supen och blir stor uti orden,
slår dig för ditt håriga bröst och blir en man från hår till häl.
\newpage
Ja, nu skall vi ut och härja,
supa och slåss och svärja,
bränna röda stugor, slå små barn och säga fula ord:
<i>Fy fan!</i>
Med blod skall vi stäppen färga,
nu änteligen lär jag
kunna dra nån verklig nytta av
min Hermodskurs i mord.

Hurra, nu ska vi äntligen få röra på benen
hela stammen jublar och det spritter i grenen.
Tänk, att än en gång få spränga fram på Brunte i galopp.
Din doft, o kära Brunte, är trots brist i hygienen
för en vild mongol minst lika ljuv som syrenen.
Tänk att på din rygg få rida runt i stan och spela topp.

Ja, nu skall vi ut och härja...

Ja, mordbränder är klämmiga, ta fram fotogenen,
eftersläckningen tillhör just de fenomenen
inom brandmansyrket som jag tycker det är nån nytta med.
Jag målar för mitt inre upp den härliga scenen:
blodrött mitt i brandgult. Ej ens prins Eugen en
lika mustig vy kan måla, ens om han målade med sked. 

Ja, nu skall vi ut och härja...}

\songinfo{Ur Lundaspexet "Djinghis Khan" 1954.}

\newpage 

%---------------------------------------------------------
\song{Spritbolaget}      
\melo{Melodi: Snickerboa}
\tags{Spritbolaget, Emils spritvisa}

\songtext{Till spritbolaget ränner jag
och bankar på dess port.
Jag vill ha nåt som bränner bra
och gör mig skitfull fort.
Expediten sade goddag,
hur gammal kan min herre va?
Har du nåt leg, ditt fula drägg,
kom hit igen när du fått skägg.

Men detta var ju inte bra,
jag vill bli full i kväll.
Då plötsligt en idé fick jag,
de har ju sprit på Shell.
Flaskorna de stod där på rad,
så nu kan jag bli full och glad. 
Den röda drycken rinner ner...
<i>Drycken intages.</i>
...nu kan jag inte se nåt mer.}

\songinfo{DISK KM sjunger “Jag kan lära dig C” efter “Spritbolaget” och börjar på 4:e raden.}
\newpage

%---------------------------------------------------------
\song{Bamsesången}        
\melo{Melodi: Signaturmelodin till Bamse}
\author{Text: D-LTH, Sångarstriden 1987}

\songtext{Jag skall festa, ta det lugnt med spriten.
Ha det roligt utan att va' full.
Inte krypa runt med festeliten,
ta det varligt för min egen skull.

Först en öl i torra strupen,
efter det så kommer supen,
i med vinet, ner med punschen,
sist en groggbuffé.

Jag är skitfull, däckar först av alla,
missar festen men vad gör väl de'.
Blandar hejdlöst öl och gammal filmjölk,
kastar upp på bordsdamen bredve'.

Först en öl…

Spyan rinner ner för ylleslipsen
Raviolin torkar i mitt hår
Vem har lagt mig här i pissoaren
Vems är gaffeln i mitt högra lår}

\songinfo{Sista fyra raderna sjungs ofta flera gånger till andra välkända melodier såsom Barbie Girl och Tomtarnas julnatt}

\newpage
%---------------------------------------------------------
\song{Uti min mage}       
\melo{Melodi:  Uti vår hage}

\songtext{
Uti min mage en längtan mig tär,
kom hjärtans fröjd.
Där råder en hunger som ropar så här:
kom kryddsill och kall potatis,
kom brännvin och quantum satis,
kom allt som kan drickas,
kom hjärtans fröjd.

Uti min mage där växa begär,
kom hjärtans kär.
Vill du mig något så har jag det där.
Kom Renat och Aqua Vitae,
kom OP och allt vad sprit e',
kom ljuva Genever,
kom Överste.

Uti mitt hjärta en längtan mig tär,
kom hjärtans fröjd.
Där råder en hunger som ropar så här:
kom famnande lena armar,
kom läppar och sköna barmar,
kom fagraste kvinnor,
kom hjärtans fröjd.}

\newpage

%---------------------------------------------------------
\song{Det var länge sen}    
\melo{Melodi: Det var länge sen jag plocka' några blommor}

\songtext{
Det var länge sen jag plocka' några tentor.
Det var länge sen jag tog några poäng.
Det var länge sen jag handla' på Systemet.
Det var länge sen jag fick en grabb i säng.
Men å andra sidan bränner jag ju hemma,
och klarar kärleken alldeles för mig själv.
Det var länge sen jag plocka' några tentor,
men å andra sidan går de om igen. }

%-------------------------------------------------------------
\song{Jag var full en gång}       
\melo{Melodi: Flottarkärlek}

\songtext{
Jag var full en gång för länge sen,
på knäna kröp jag hem.
Varje dike var för mig ett vilohem.
I varje hörn och varje vrå
hade jag en liten vän,
ifrån renat upp till 96 procent, hemmabränt.

Jag var full en gång för länge sen,
på knäna kröp jag hem,
och i sällskap hade jag en elefant.
Elefanten spruta' vatten,
men jag trodde det var vin,
sedan dess har alla kallat mig för svin, mera vin!
\newpage
Jag var full en gång för länge sen
på knäna kröp jag hem,
och i sällskap hade jag en elefant.
Elefanten spruta' vatten,
men jag trodde det var öl,
sedan dess har alla kallat mig för knöl, mera öl!

Jag var full en gång för länge sen,
på knäna kröp jag hem,
och i sällskap hade jag en elefant.
Elefanten spruta' vatten,
men jag trodde det var sprit
sedan dess har alla kallat mig för skit, mera sprit!}

%-------------------------------------------
\song{Var rädd om din fyrfota vän}     
\melochtext{}

\songtext{Var rädd om din fyrfota vän,
för en anka kan vara dess mamma.
Som simmar omkring i en damm
just när solen tittar fram,
och nu tror du att visan är slut
och det är den!
(Inte!)}

\songinfo{Kan upprepas nästan hur många gånger som helst...}
\newpage
%-------------------------------------------

\song{Dance macabre}     
\melochtext{Melodi: Vårvindar friska}

\songtext{Runt kring vår stuga, smådjävlar sluga,
dansa så tyst med bockfot och svans.
Varulvar yla, isande kyla,
sveper i dimman fanstygens dans.
Bäva o broder, lyssna och hör,
vrålen från gast, som osalig dör.
Satan han skrattar, flaskan han fattar,
super tills dagen gryr.

Gastar och spöken, skymtar i kröken,
dödingar släpa ruttnande lik.
Benrangel skramla, spökhänder famla,
kväva din strupes rosslande skrik.
Helvetes alla fasor släpps loss.
Fan rider här med hela sin tross.
Göm dig i stugan, du har fått flugan.
Dille det blir din lott.}

\newpage
%---------------------------------------------------------

\song{Född i Norge}   
\melo{Melodi: Oh my darling Clementine}

\songtext{
Född i Norge, bor i Sverige,
Danmark är mitt fosterland,
talar spanska som en jude,
för jag är en engelsman.

Full idag och full imorgon,
så ser livet ut för mig.
Jag ska aldrig svika flaskan,
jag skall aldrig gifta mig.

Och på min gravsten, på min gravsten,
ska det ristas på latin
att i kistan vilar stoftet
av ett jävla fyllesvin

Och alla maskar, alla maskar
de skall krypa i min kropp.
Och de ska bli så djävla fulla
att de aldrig hittar opp.}

\newpage

%-------------------------------------------

\song{Kalmarevisan}
\melochtext{}
\songtext{Uti Kalmare stad
ja, där finns det ingen kvast
förrän lördagen.

<b><i>Hej dick...</b></i>
...hej dack!

<b><i>Jag slog i...</b></i>
...och vi drack,

<b><i>Hej dickom dickom dack!</b></i>
Hej dickom dickom dack!

För uti Kalmare stad, 
ja, där finns det ingen kvast
förrän lördagen.

\leftrepeat <b><i>När som bonden kommer hem</b></i>
kommer bondekvinnan ut \rightrepeat
är så stor i sin trut.

<b><i>Hej dick...</b></i> (osv.)

\leftrepeat <b><i>Var är pengarna du fått?</b></i>
Jo, dom har jag supit opp \rightrepeat
uppå Kalmare slott.

<b><i>Hej dick...</b></i> (osv.)

\leftrepeat <b><i>Jag ska mäla dej an</b></i>
för vår kronbefallningsman \rightrepeat
och du ska få skam.

<b><i>Hej dick...</b></i> (osv.)

\leftrepeat <b><i>Kronbefallningsmannen vår</b></i>
satt på krogen igår \rightrepeat
och var full som ett får.

<b><i>Hej dick...</b></i> (osv.)


<b><i>Kalmariten:</b></i>
\leftrepeat <b><i>När jag ser en kalmarit</b></i>
Tar jag fram min dynamit \rightrepeat
Och spränger honom till skit!

<b><i>Hej dick...</b></i> (osv.)}

\songinfo{Det kursiva sjunges av en försångare.}


\newpage

%-------------------------------------------
\song{Balladen om den onyktre}       
\melo{Melodi: När månen vandrar på fästet blå}

\songtext{När jag är fuller då är jag glad,
fan vet om jag ej är vacker.
Jag vandrar kring i vår lilla stad,
ibland lyxhus och baracker.
Jag sjunger ljuvligt en serenad,
det gör jag bara när jag är glad
och full och vacker, och full och vacker.

När jag är fuller då är jag stark,
fan vet om jag ej är modig.
Då kan jag slå vem som helst i mark,
så han blir trasig och blodig.
Jag välter träden i våran park,
det gör jag bara när jag är stark
och full och modig, och full och modig.

När jag är fuller då är jag rik,
fan vet om jag ej är snille.
Och dör jag blir jag ett vackert lik,
begravs med gravöl och gille.
I himlen möts jag av hornmusik,
det gör man bara när man är rik
och är ett snille, och är ett snille.


Men när jag vaknar upp nästa dag,
uppå ett enkelrum med galler.
Då känner jag mig så rysligt svag,
och hatar bråk och kravaller.
min mage krånglar och är ur lag,
nog fan så vet jag att jag idag
är bakom galler, är bakom galler.


<b><i>Damernas extravers: Text av Emma Wibom</b></i>

När jag är fuller då är jag snygg,
fan vet om jag ej är stilig.
Då kan jag få vem som helst på rygg,
då kan jag få vemhelst villig!
Ja, alla gossarna får jag omkull,
det får jag bara när jag är full
och snygg och stilig, och snygg och stilig.}

%---------------------------------------------------------
\song{Eno}     
\melo{Melodi: Staffan stalledräng}

\songtext{
Eno är en masochist,
vi slår honom så gärna.
Motorsåg och giftig kvist,
allt för den sjuka hjärna.
Inga skador synes än,
spikarna i huvudet de blänka.}

%-------------------------------------------
\song{Moralvisa}     
\melo{Melodi: Vem kan segla} 

\songtext{Den som dricker mer än han tål,
strax runt badrummet crawlar,
i sitt surplus av får i kål,
bland roll-onnar och tvålar.}

%-------------------------------------------
\song{Raj, raj}        
\melo{Melodi: Elvira Madigan}

\songtext{Om vi inga texter kunna
sjunga vi blott dessa ord:
Raj, raj, raj, raj, raj, raj, raj, raj,
raj, raj, raj, raj kring vårt bord.}

%------------------------------------------
\song{Solen}        
\melo{Melodi: Camptown Races}

\songtext{Solen den går upp och ner, doda doda.
Jag skall aldrig supa mer, hej doda dej.
Hej doda dej, hej doda dej.
Jag skall aldrig supa mer, hej doda dej.

Men detta det var inte sant, doda doda.
I morgon gör jag likadant, hej doda dej.
Hej doda dej, hej doda dej.
I morgon gör jag likadant, hej doda dej.}

\newpage

%-------------------------------------------
\song{Lyft ditt välförsedda glas}     
\melo{Melodi: Ding Dong Merrily on High}

\songtext{
Lyft ditt välförsedda glas
det är en ljuvlig börda,
nu har grabbarna kalas,
och vi skall segern skörda!
Ding, dinge-dinge-ding
dinge-dinge-ding
dinge-dinge-ding, dong-dong,
i morgon är det lördag.

Lyft nu glaset till din mun,
se, döden på dig väntar!
Nu har grabbarna kalas,
hör, liemannen flämtar!
Ding, dinge-dinge-ding
dinge-dinge-ding
dinge-dinge-ding, dong-dong.
Begravningsklockor klämtar.}



%-------------------------------------------
\song{Störthärligt full}       
\melo{Melodi: Fat Mammy Brown}

\songtext{Nu har alla lämnat festen
och jag sitter ensam kvar
ibland groggar, pilsnerflaskor
i en sönderslagen bar.
Sista pilsnerflaskan tog jag
till min frukost klockan fem
och nu sitter jag och väntar
på att få bli buren hem.

För jag är störthärligt full
och jag ramlar mest omkull.
Jag ser skära elefanter
som har jättekonstig ull.
Ja, jag är störthärligt full
och jag ramlar mest omkull.
Det är präktigt, härligt,
supa och va' full.

Ifrån festen minns jag inget,
men mitt öga blev visst blått.
Och det måste jag ha fått
när någon kastat en karott
full med vispgrädde och fimpar
och en okammad peruk,
eller också när jag stod
i moraklockan och var sjuk.

För jag är...

Nästa morgon när jag vaknar
med en bergsborr i min kropp.
Sandpapper på tungan
och jag vill ej stiga opp.
Mina armar dom känns tunga
och min näsa den är sne'.
Så jag raglar ut till köket
för en återställare.

För jag är...}

%-------------------------------------------

\song{I ett hus vid skogens slut}
\melo{}

\songtext{I ett hus vid skogens slut,
liten tomte tittar ut.
Haren skuttar fram så fort,
klappar på dess port.
Hjälp ack, hjälp ack, hjälp du mig,
annars skjuter jägarn mig.
Kom, ja, kom i stugan in,
räck mig handen din.

En köttkorv är nog
både nyttig och god,
men såsen gör magen så rund...
Nej, grönt ska det va',
tänk, så skönt att få ha
med gurka en ljuv herdestund!

Jag har testat kiwi,
men den var för rivig,
nej, ta av bananen dess skal!
För då har jag känt
att dess friktionsko'ff'cient
har blivit för mig optima-al!

Banan, vilken grej!
Bra för mig som är tjej!
Jag är ju ve-ge-tarian!
Med skal eller ej,
jag vill ha den i mig.
Sakta det går med banan...
Sakta det går med banan...}

\songinfo{Lantmäterisektionens bidrag i Sångartäflan 1987.}

%-------------------------------------------

\song{A long time ago}       
\melo{Melodi: Schuberts Marche Militaire}

\songtext{
A long time ago in a small town in Germany
there was a shoemaker, Shoemaker was his name.
He could play the violin, violin, violin
he could play the violin, vio-violin.

<i>Spela fiol.</i>

A long time ago...
He could play the trombone...

A long time ago...
He could play the piccolo...

A long time ago...
He could play the el-guitar...

A long time ago...
He could play the bloody fool...}

%-------------------------------------------
\song{Nu blotas det}      
\melo{Melodi: Nu grönskar det}

\songtext{Nu blotas det i stad och land,  
nu är midsommartid. 
Dräp ko, dräp häst, det är jättefest 
se blod överallt och bredvid.
Vart djur som dör ger bättre skörd, 
se där har vi ju ett svin! 
Ja, dräp nu allt, dräp ko och galt 
och taxar och braxar och bin.

Rakt fram mot offrets gråa bord 
vi glatt vår kossa styr.
Vi slaktar Bambi och Stampe med, 
se upp så att Bamse ej flyr.
I gyllne dryck så skålar vi 
när vi står i det blod som vi skvätt. 
Det där det var gott, får man mer om man ber?
Ja, punsch uti blodet känns rätt.}

\songinfo{Ur juristspexet Ansgar 2003}

\newpage 

%-------------------------------------------

\song{Lille Olle}    
\melo{Melodi: Katjuscha}
\author{Text: Calle Isaksson}

\songtext{
Lille Olle skulle gå på disco,
men han hade inte någon sprit.
Lille Olle skaffa' lite hembränt,
lille Olle gick då på en nit.

La lalla la la la...

Lille Olle skulle börja festa,
spriten blandade han ut med Mer.
Lille Olle drack upp hela bålen,
lille Olle ser nu inte mer.

La lalla la la la...

Lille Olle skaffade en ledhund,
den var ful, men även ganska trind.
Olles ledhund drack upp femton flaskor,
Olles ledhund är nu också blind.

La lalla la la la...

Lille Olle började med droger,
blandade sin LSD med juice.
Lille Olles hjärna står i lågor,
lille Olle dog av överdos.

La lalla la la la...

Lille Olle sitter nu i himlen,
festa kan man även göra där.
Lille Olle skaffade en ölback,
capsar nu med Gud och Sankte Per.

La lalla la la la...}

\songinfo{Skrevs 1991 då textförfattaren gick D-linjen på LiTH.} 

%-------------------------------------------
\song{En liten blå förgätmigej}
\melochtext{}

\songtext{Hur gärna ville jag ej vara,
en liten blå förgätmigej, 
en liten blå förgätmigej. 
Då skulle jag för dig förklara, 
hur innerligt jag älskar dig}

\songinfo{Sjungs i slutet av sittningar som tack till serveringspersonalen.}

\newpage 

%-------------------------------------------
\song{Det var i vår ungdoms fagraste vår}
\melochtext{}

\songtext{Det där det gjorde han/hon/hen/de fan så bra,
en skål uti botten för honom/henne/hen/de (nu) vi ta.
Hugg i och dra - hej!
Hugg i och dra - hej!
En skål uti botten för honom/henne/hen/dem (nu) vi ta.
Alla så dricka vi nu [namn] till,

<i>och [namn] säger inte nej därtill.</i>

Det var i vår ungdoms fagraste vår,
vi drack varandra till, och vi sade gutår!}

\songinfo{Sjunges som tack för något berömvärt. Det kursiva sjunges enbart av den som sången riktas till och kan med fördel bytas ut mot något annat, som gärna får rimma på ”till”. Gutår är en gammelsvensk synonym till skål. Undvik att skåla dubbelt.}


%-------------------------------------------



\song{Hej på er vänner alla}
\melo{}
\author{}
\tags{}
\songinfo{}

\songtext{Hej på er vänner alla,
ja vi ska supa tills dess vi falla,
och brännvinslitern, den är för liten,
den är för liten för oss alla!

Och en gång när jag är döder
och lagd mellan tvenne vänner.
Begrav mig, begrav mig
i en brännvinskällare på Söder.

Och på min gravsten skall det stå ristat
med tvenne små enkla rader:
''Här vilar det en fylletrogen,
som alltid var så glad och goder.
Här vilar det en fylletrogen,
som alltid var så glad och goder.''}




%-------------------------------------------------------------


\song{Gasqueljäsen}
\melo{Melodi: La Marseillaise}
\author{}
\tags{FRANKRIKE}
\songinfo{Fysikalen Kristina 1982}

\songtext{Vi dricker öl, vi dricker alkohol,
vi dricker mer än vad vi tål!
Det var kaos och dimmigt i Lützen,
men här är det värre ändå!
Här har bordet just börjat att gå,
och denna fest helt utan lik é
så här är det inte så dött.
Min lever har sitt öde mött!
Skålar gör vi för vårt svenska rike!
Vi super med kultur,
fast vi vet inte hur,

||: Drick ur, drick ur
tills det tar slut
1: Vi dricker med bravur! :||
2: Vi super med kultur!}




%-------------------------------------------------------------


\song{Mattevisan}
\melo{Melodi: Jag är en liten undulat}
\author{Text: Jan-Ola Vensson, F}
\tags{}
\songinfo{}

\songtext{Jag är en liten teknolog
som har så helvetes svårt,
med all den matten,
med all den matten,
jag måste läsa.

Jag vill ha öl och billigt vin.
Kan kalla mig fyllesvin.
Ja, derivata och integraler
de kan ni glömma!}




%-------------------------------------------------------------


\song{37:an}
\melo{Melodi: 34:an}
\author{}
\tags{}
\songinfo{Ur Lunds Fysikteknologers sångbok 1999}

\songtext{Jag har druckit många punschar,
blandat grogg i alla år
svept en herrans massa cognac
och vält 100 000 får - VA?!
Fått betongkeps utav rödvin,
haft likör som måltidsdryck
sabbat Chivas med en Cola,
halsat folköl med en knyck.

Men nu e' det slut på halvmesyrer
nu ska blannevattnet bort,
här ska rationaliseras, proceduren göras kort.
Fina spriten flödar bäst i strupen utan fint manér,
så helt utan krusiduller går 37:an i magen ner!}




%-------------------------------------------------------------
\song{Frankrike dricks det}
\melo{Melodi: Te Deum laudeamus}

\songtext{} I frankrike dricks det  viner.
Mär tyskarna dricker ölunderbart det mår.
Men svensken som dricker, svin är
oss svin emellan:
Tag en tår!

%-------------------------------------------------------------
\song{etanol}
\melo{Blinka lilla stjärna}

\songtext{}Köra runt i bil går bra
full och galen, go och glad
Tanken full med etanol
till polisen säga skål,
luktar det av sprit, jaha
det är bilen, inte jag