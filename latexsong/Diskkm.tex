

% Headertext
\headertext{DISK KM}

%---------------------------------------------------------

\song{DISKs Nationalsång} \\       
\melo{Melodi: Rysslands Nationalsång}

\songtext{Vårt namn är D I S K} \\  
från Kista vi äro,\\  
Vi kan vara snälla eller ställa till besvär\\  
Men drabbats ni har av en olycka stor\\  
för ikväll ska vi supa och bedriva hor.  

Ni längtar nog,\\  
till dess vi åker bort igen,\\  
med djurtransporten till Sibirien.\\  
Men mest utav allt så längtar ni till,\\  
att med oss få grova så mycket ni vill.\\  
vårt namn är DISK!  

\songinfo{* Sjunges stående med handen på hjärtat och utan huvudbonad.}\\

\newpage 

%-------------------------------------------

\song{Jag kan lära dig C} \\       
\melo{Melodi: En helt ny värld}
\author{Text: Arian och Sixten }

\songtext{}Låt mig visa dig kod\\
Vacker, effektiv, härlig\\
Säg om du ska va ärlig, \\
har du drömt om den ibland?

Jag kan lära dig C\\
Egen minneshantering\\
Pekare och funktioner \\ 
om du kodar här med mig

Hello World!\\
Här får vi göra som vi vill\\
Koden regerar här!\\
C och lär!\\
Kompilatorn är vår vän\\
En helt ny värld\\
Utan nån \textit{\textbf{jävla MDI}}\\
Inget designertjat, så underbart\\
Att koda i en helt ny värld, med dig

\songinfo{* Framfördes på julfesten 2011. Ett tag var den en stående omstart och senare blev känd som låten där samtliga närvarande DISK KMare ska bli utburna. Sedan har den gått vidare till att bli en typisk låt att sjunga i konjunktion med Spritbolaget.}\\

\newpage
%-------------------------------------------

\song{Onanera} \\       
\melo{Melodi: Hallelujah, (L Cohen)}
\author{Text: Sober}

\songtext{Jag brukade vara nere förr}\\
tills jag upptäckte min bakdörr\\
Nu vill jag ju bara smeka mera\\
Då hörde jag en inre röst\\
nu saknar jag ju inga bröst\\
Den viskade och beordra’ \\
”Onanera!” \\
Onanera, onanera\\
Onanera, onanera

Jag är nästan som en datalog\\
Det var rätt länge sen jag låg\\
Men jag saknar det faktiskt inte mera\\
Jag satte mig vid datorns skärm\\
och drog servetten ur min ärm\\
Och började så smått att onanera\\
Onanera…

Det finns ett drag i varje man\\
att röra nåt så fort han kan\\
Och lyxrunk, det krävs bara att planera\\
Och snälla, det finns ingen gräns!\\
Om man inte har impotens\\
Men man kan ändå andra onanera!\\
Onanera…

\songinfo{* Skrevs på DISK KMs externsittning den 7e april 2015 ett spex om onani.}

%-------------------------------------------

\song{Sång till DISK KM} \\       
\melo{Melodi: Sång till friheten}
\author{Text: Ida Wellner}

\songtext{Vi är det finaste vi vet},\\
vi är det vackraste i världen.\\
Vi är så svarta med revärer utav guld,\\
här från Sibirien regerar vi vårt välde.\\
DISK KM är vårt stolta namn.\\
Marskalk - en titel som ska äras!\\
Svabbmeistrar - gamla klubbor, vraken - veteraner,\\
Razor är vår gud och trasorna hans pungsvett.\\
Vi är det finaste vi vet!

\songinfo{* Framfördes som klubbmästarspex på en internfest under trasperioden hösten 2000.}\\

%-------------------------------------------

\song{Tappra trasor} \\
\melo{Melodi: Internationalen}
\author{Text: Sara von Knorring }

\songtext{Det är en skara tappra trasor},\\
som genomlidit stor misär.\\
De sitter här och känner fasor,\\
anar ondska och onda begär.

De kämpar för att kunna orden\\
att inte missa någon lag.\\
Men trasans onda lott på jorden\\
ger varken hopp eller sköna behag.
\newpage
Vi har hotat med donken,\\
vi har dunkat med hand,\\
vi har handskats med lorten,\\
vi har lortat när vi kan.

Vi har kunskap som skrämmer,\\
vi har skrämt dem till lik,\\
vi har liknat var trasa\\
vid en avgrundsdjup tragik. Tragik, tragik!

När vraken slabbat ner en trasa,\\
så går den knappt att vrida ur.\\
Två ögon stirrar då med fasa,\\
kan det vara min tur nu?

Nu är det dags att visa vägen\\
att supa in de få som tål.\\
På slutet vinner alltid trägen,\\
starkare än alkohol.

Vi har hotat med donken...

\songinfo{* Skrevs till insupet hösten 2001.}\\

\newpage 

%-------------------------------------------

\song{Donkkanon} \\       
\melo{Melodi: Punschkanon}
\author{Text: Sara von Knorring }

\songtext{} 
\textbf{\textit{Herrarna:}}\\
\leftrepeat Donk, donk, donk, donk, donk, donk, ge dem mera \rightrepeat

\textbf{\textit{Damerna:}}\\
Vi tager vad vi haver och blandar till en smet,\\
det luktar gammal flodhäst, ser ut som monsterklet.\\
\leftrepeat Åh donk, åh donk, åh gammal hederlig donk. \rightrepeat\\
Om trasan inte lyssnar och lär sig det den hör\\
då ljuder visselpipan och röster höjs i kör.\\
\leftrepeat Åh donk, åh donk, åh ge dem mera donk. \rightrepeat\\
En tjock och klibbig kyla i trasans strupe sprids,\\
magen exploderar och verkligheten vrids.\\
\leftrepeat Åh donk, åh donk, åh ännu mera donk. \rightrepeat\\
En lättad marskalk flinar, betraktar trasans fel,\\
men här går ingen säker, det kallas "passivt spel"!\\
\leftrepeat Åh donk, åh donk, åh ännu mera donk. \rightrepeat\\
När vraken supit ut sig, men ingen trasa fått,\\
då gosar de med donken och tröstar sig så smått.\\
\leftrepeat Åh donk, åh donk, åh söta rara donk. \rightrepeat

\songinfo{* Skrevs till en internfest hösten 2001.}\\

\newpage

%-------------------------------------------

\song{Spexkanon} \\       
\melo{Melodi: Punschkanon}
\author{Text: Rasmus Larsson, Osborne von Waldegg, m.fl.}

\songtext{} 
\textbf{\textit{Herrarna:}}\\
\leftrepeat Spex, spex, spex, spex, spex, spex, ännu mera \rightrepeat

\textbf{\textit{Damerna:}}\\
Vi sitter på vårt förkör å spånar jävligt hårt,\\
men skriva visor utan snusk, det är så jävla svårt.\\
\leftrepeat Åh spex, åh spex, vad hände med vårt spex? \rightrepeat\\
När vraken skriver spexet då blir det inget gjort,\\
vi sätter oss på tåget, då går allt jävligt fort.\\
\leftrepeat Åh spex, åh spex, vad hände med vårt spex? \rightrepeat\\
Idéer om Bin Ladin, de ratas ganska kvickt,\\
precis som våran fist-sång å annat jävligt sickt.\\
\leftrepeat Åh spex, åh spex, det kanske blev ett spex? \rightrepeat\\
Så nu vi står på festen å sjunger ganska bra,\\
å vi är ganska roliga, här saknas det en rad.\\
\leftrepeat Åh spex, åh spex, det blev ändå ett spex! \rightrepeat

\songinfo{* Skrevs under förköret till en marskalkssittning hos Föreningen Ekonomerna våren 2002.}

\newpage 

%-------------------------------------------

\song{Trasig marskalk} \\       
\melo{Melodi: Jag är fattig bonddräng}
\author{Text: Martin Johnsson}

\songtext{Jag är trasig marskalk men jag lever ändå}.\\
Pubar går och kommer medan jag knogar på.\\
Svabbar, går och tömmer, sköljer, torkar och bär.\\
Står med Falcon-tappen, häller öl till vår här.

Jag är trasig marskalk och har capsat i grus.\\
Uti vida världen har jag skaffat ett rus.\\
Hånglat, grovat, horat, har jag också förstås.\\
Bytt en massa märken, så har jag festat loss.

Så gått alla pubar, fester, roligt och kul.\\
Med de alla arren och med andra haft strul.\\
Jobba, slita, svettas, ja terminerna ut.\\
Åren som en marskalk, de har nu tagit slut.

Men då säger Razor: ”Trasig marskalk, kom hit!\\
Jag har sett din strävan och ditt eviga slit.\\
Därför, trasig marskalk, är du välkommen här.\\
Därför, trasig marskalk, skall du vara mig när.”

Och jag, trasig marskalk, står så still inför Gud.\\
Iklädd svart och guld i våran heliga skrud.\\
”Nu du”, säger Razor , ”är ditt arbete slut.\\
Nu du, trasig marskalk, nu får du vila ut.”

\songinfo{* Skrevs till Martinis vrakning på internfesten 1/10 2005. Framfördes av Martini själv som enligt egen utsago ”fick lite damm i ögat”.} \\

%-------------------------------------------

\song{Blind men full} \\       
\melo{Melodi: Vi gå över daggstänkta berg}
\author{Text: Janne Weijnitz }

\songtext{Vi dricker den sprit som vi får, fallera}.\\
Av finkel så känner vi små spår, fallera,\\
och sorger har vi inga\\
där vi blinda går och vingla.\\
För vi har ju ett rus skaffat oss, fallera.\\

%-------------------------------------------

\song{Längden har visst betydelse} \\       
\melo{Melodi: I Apladalen i Värnamo}
\author{Text: CAlle Almquist}

\songtext{På långa visor, epos, ballader},\\
med sina ändlösa antal rader,\\
jag blivit trött, ge mig nu en kort!\\
Min nubbe står här och dunstar bort!

\songinfo{* Framfördes till kraftigt jubel på kräftskivan 26/8 2006.}\\
%-------------------------------------------

\song{Jag är en full person} \\       
\melo{Melodi: Mer jul}

\songtext{} 
Jag är en full person, som tappat skon,\\ 
stupfull och enerverad.\\ 
Jag har nubbe och sill, och det ska lite till,\\ 
innan jag tar mycket mera.
\newpage
Men jag har en donk som räcker rätt långt,\\ 
ja, den räcker till varje flaska.\\ 
När maten är slut, och klubban bärs ut,\\ 
och vi andra börjar gasqua. 

\leftrepeat Jag vill ha, mer sprit\\ 
Ge mig, mer sprit. \rightrepeat

Tusen flaskor som klirrar,\\ 
groggar så långt jag ser,\\ 
små röda ögon som tindrar,\\ 
vill jag ha fler.

En student glöms bort, om han bara leker sport,\\ 
eller tittar på konstiga filmer.\\ 
Nej, ge mig ovvar med stil, och mängder med fil\\ 
och donk att blanda ut med. 

En sektion som drar in cash och inte dricker bärs,\\ 
en Sekt som pillar navel.\\ 
Men KM kan sprit, när vi får invit,\\ 
står barskåpet på vid gavel.

Jag vill ha..

\songinfo{* Denna sång skrevs gissningsvis någon gång runt 1999-2000. Tyvärr kommer ingen längre ihåg vem som skrev den eller i vilket sammanhang.} \\

\newpage 

%-------------------------------------------

\song{Disciplinsång} \\       
\melo{Melodi: Rövarna i Kamomilla stad}
\author{Text: Karin Holm och Sara von Knorring}

\songtext{Nu ska vi lära disciplin},\\
nu är vi här och festar.\\
Men vi är jäkla fyllesvin\\
som tillvaron förpestar.\\
Det händer nog att vi ibland\\
vill ha en saftig tår på tand.\\
Nu tänker vi stilla vårt onda begär,\\
både marskalk å trasa å vrak minsann.\\
Pudumpumpum...\\
En varning är nog på sin plats\\
när DISK är här i huset.\\
Då blir det både öl och caps\\
vi stå för värsta buset.\\
Så hissa upp er röda flagg,\\
håll armen hårt omkring ert ragg.\\
Nu tänker vi stilla vårt onda begär,\\
både marskalk å trasa å vrak minsann.

\songinfo{* Skrevs till en marskalkssittning på Lärarhögskolan hösten 2001, temat var disciplin.}\\

\newpage 

%-------------------------------------------

\song{En enkel sång om studentlivet} \\       
\melo{Melodi: Jag fångade en räv}

\songtext{Jag skulle åka hiss idag},\\ 
från 3:an upp till 5:an.\\ 
Men hissfan har ett eget liv,\\ 
jag hamnade i källar'n.\\ 
Hissfan fungerar ej\\ 
så i trappa blir det spring.\\ 
Vad gör jag på DSV\\ 
när allting kommer kring?\\ 
Pascal är ingen match sa jag\\ 
och skulle imponera.\\ 
Men ångra vad jag sa fick jag\\ 
när jag skulle kompilera.\\ 
Records hit och errors dit,\\ 
jag fattar ingenting.\\ 
Vad gör jag på DSV\\ 
när allting kommer kring?\\ 
Vi pluggar här på DSV\\ 
och tror vi valt det rätta,\\ 
för kurserna på KTH\\ 
vi tyckte var för lätta.\\ 
KTH har låg nivå,\\ 
dom lär sig ingenting.\\ 
Vi har kul på DSV\\ 
när allting kommer kring.

\songinfo{* Skriven av DISKtrasorna 1995.}

\newpage 

%-------------------------------------------

\song{Nu krökas det} \\       
\melo{Melodi: Nu grönskar det}
\author{Text: förmodligen Rasmus Larsson}

\songtext{} 
Nu krökas det i Kistas famn,\\
nu supas öl och sprit.\\
Kom med, kom med på fylleslag\\
och ramla hit och dit.\\
Var punsch är som en gyllne tår\\
som svalkar ditt begär.\\
Så skål min vän, sup ner dig nu\\
ty kvällen den är här.

\songinfo{* Rasmus hittade sången uppskriven på baksidan av "Dagordning 02-05-17 Extra årsmöte" men det är oklart om han framförde den då, eller om den framförts överhuvudtaget.}\\

%-------------------------------------------

\song{Kort svinvisa} \\       
\melo{Melodi: Tomten, jag vill ha en riktig jul (refrängen)}
\author{Text: ”Ärkesvinet” Janne Weijnitz }

\songtext{Pappa, jag vill bli ett ärkesvin},\\
böka och stöka, kröka och pöka.\\
Pappa, jag vill bli ett ärkesvin,\\
ett sånt som man är när man har roligt, skål!

\songinfo{* Skriven av Janne Weijnitz 1993, bland DISK KM:s första klubbmästare}\\

\newpage 

%-------------------------------------------

\song{Sjörövarsången} \\       
\melo{Melodi: Rövarna i Kamomilla stad}
\author{Text: Martin Johnsson}

\songtext{Ja, vi skall ut på sjöslag nu},\\
vi kapar oss en Silja.\\
Vi grovar loss på haven sju,\\
dräggar med de som vilja.\\
Med hög procent blir grogg rejäl.\\
En finsk och rolig skylt vi stjäl\\
Vi tax-freen belägrar\\
och fulla vi lägrar\\
kaptenen och styrman och Silja-säl.\\
Pudumpumpum…\\
Ja, capsa i urinprovsmugg.\\
I cirkeln vi regerar.\\
Sen bryr vi oss ej ens ett dugg,\\
vi tappen ockuperar.\\
I färjans matsal ligger vi,\\
och finska ölen flödar i.\\
Den lindrar besvären\\
men väcker begären\\
för styrman, kapten och en Silja-säl.

\songinfo{* Varje höst brukar DISK KM vara med på den jättelika studentfesten Sjöslaget. Den äger vanligen rum på en av Silja Lines finlandsbåtar där spriten är billig och maten på buffén aldrig tar slut för den som har aptiten i behåll på hemresan. Sången framfördes av Spexmesteriet på den lika traditionella sillunchen innan resan 10/10 2004.}

\newpage 

%-------------------------------------------

\song{Min värsta tid} \\       
\melo{Melodi: Vår bästa tid är nu}
\author{Text: Martin Johnsson}

\songtext{} 
Min värsta tid blir snart\\
för jag kan inte längre sitta rakt.\\
Och det känns underbart.\\
Jag har fördruckit mig på öl och vin och donk.\\
Tror nog fan att det känns,\\
aldrig att jag druckit hälften ens.\\
Men nu jag törstig är,\\
så jag får nog allt ta en till, och till, å till.

\songinfo{* Framfördes av trasa 6 till sittande, tysta ovationer på hans insup 10/5 2003.}\\

%-------------------------------------------

\song{Ge oss nå't som dövar} \\       
\melo{Melodi: I Apladalen i Värnamo}
\author{Text: CAlle Almquist}

\songtext{På denna falsksång från våra trasor},\\
med usla rim där de sjunger ”Razor”,\\
vi vill ibland inte höra alls,\\
blott få nå't dövande i vår hals.

\songinfo{* "Razor" i texten ovan uttalas alltså "rasor" och inte "rejsor". Helt tokigt! Det här är för övrigt en sång man inte vill behöva sjunga särskilt ofta.}\\

\newpage 

%-------------------------------------------

\song{Subliminalt meddelande} \\       
\melo{Melodi: Pippis sommarvisa}
\author{Text: CAlle Almquist}

\songtext{Festade med gyckel, fylla, stoj och glam}.\\
Skulle ta min cykel men hamna' bak och fram.\\
Stjärten uppå styret, det har man inget för.\\
Huden blir lätt rispad, man ser ej vart man kör.

\songinfo{* Sjöngs kanske för första gången på kräftskivan 20/8 2004.} \\

%-------------------------------------------

\song{Utnämnda av Razor} \\       
\melo{Melodi: Han har öppnat pärleporten}
\author{Text: Martin Johnsson}

\songtext{Vi av Razor är utnämnda} \\
med sin vishet djup och stor \\
han har givit oss ett ansvar \\
för att Razor på oss tror. \\
Han har öppnat våra sinnen \\
så att KM leva kvar. \\
Genom donken har han frälst oss \\
och välsignat oss han har. \\
En gång, vi var ynklig trasa. \\
Nu vrak, marskalkar, styrelsen. \\
Vi förkunnar nu de budord \\
som Razor skrev i sten. \\
Han har öppnat våra sinnen...

\newpage 

%-------------------------------------------

\song{Haiku till DISKs marskalkssittning} \\       
\author{Text: Sebastian Stureborg}

\songtext{Mulen kväll vid DISK} \\
Framtiden är mörk för mig \\
där ett ljus, Foo Bar

\songinfo{* Framfördes på marskalkssittningen 1/3 2004 av en DSV-student med ett förflutet i Lunds klubbmästerivärld.} \\


%-------------------------------------------

\song{Klappa ej djuren för fan!} \\       
\melo{Melodi: Du ska få min gamla cykel när jag dör}
\author{Text: Martin Johnsson}

\songtext{Klappa kossan på dess våta, blöta nos}.\\ 
Smek en mule om du gillar lite gos.\\ 
Vad gör lite mul- och klöv-\\ 
när man för varningsbud är döv, \\
ja, på donken köp en rätt med kossamos.

Klappa apan om du tycker den är grann.\\ 
Smeka Tarzans morsa, bara litegrann.\\ 
Men får du ett kärleksbett,\\ 
så att blodet blir till svett,\\ 
skall du veta att det stärker varje man.

Klappa fågeln om du tycker den är fin.\\
Smek en pippi, då får du ett saligt flin.\\ 
Ja, gojan den är hes,\\ 
men du är väl ingen mes,\\ 
trots att papegojan verkar lida pin.

Klappa vraket med en smutsig ovverall.\\ 
Smek ett vrak så får du vraket snart på fall.\\ 
Men kommer du för nära\\ 
kommer sjukdom dig besvära\\ 
mot vrakbaciller trasor ej står pall.

Men blir du sjuklig får du säkert religion.\\ 
Ta då nattvard varje dag, det stärker tron.\\ 
Men har du fått för mycket,\\ 
så du inte pallar trycket,\\ 
då välkomnar nog hin håle dig ditt jon.

\songinfo{* Vrakspex skrivet till insupet 29/10 2005. En nog så viktig varning i dessa tider då allehanda farsoter hotar den som klappar och smeker för mycket eller för ofta.}\\


%-------------------------------------------

\song{Ju mera bränt vi dricker} \\       
\melo{Melodi: Ju mer vi är tillsammans }
\author{Text: Martin Johnsson}

\songtext{} 
Ju mera bränt vi dricker,\\
vi dricker, vi dricker,\\
ju skönare det sticker,\\
i halsen sticker rent.\\
För hellre rent, än alltför lent,\\
och ännu är det ej för sent.\\
Så mera bränt vi dricker,\\
tills levern sprängs av rent.\\

\newpage

%-------------------------------------------

\song{En pestig kräfta} \\       
\melo{Melodi: I Apladalen i Värnamo}
\author{Text: Martin Johnsson}

\songtext{En gång i somras jag skulle bada}, \\
så ut i sjön började jag vada. \\
Nu vi berätta vad hände då,\\
att det är sant, jo! Skall ni förstå!\\
Hörru du, ja, jag bada naken,\\
i mässing bara, visa' jag baken.\\
Denna kväll då var vattnet kallt\\
till följd av detta så krympte allt.\\
Men plötsligt när jag i vattnet simma,\\
det gjorde ont så jag nästan svimma.\\
En kräfta klämde fast på min snopp,\\
å jag sa "Dig skall jag äta opp!"\\
På min task jag tror kräftan glodde\\
"Det är en mask!" var nog det han trodde.\\
Så nu jag fiskat på detta sätt\\
alla kräftor till dagens rätt.\\
Men kräftor köpas skall i butiken\\
för hör på detta själva tragiken.\\
Att jag blev vanställd, det må va' hänt\\
men det var synd jag blev impotent!

\songinfo{* Framfördes av Spexmesteriet på KM:s årliga kräftskiva 20/8 2005.
}

\newpage

%-------------------------------------------
\song{Uti vår Foo Bar} \\       
\melo{Melodi: Uti vår hage}
\author{Text: Britney}

\songtext{} 
Uti vår Foo Bar där rinner Zubr\\
Vårt ljuva guld\\
Vill du mig någe', så träffas vi där!\\
Drick lager och goa groggar,\\
Drick suröl och massa fulsprit,\\
Ät gyllene tosten, drick Foo Bars guld!

Fagra marskalkar där bjuda på öl.\\
Drick Foo Bars guld\\
Vill du, så får du ett märke av mig!\\
Drick starksprit och punsch o' nubbe,\\
Drick Whiskey som luktar gubbe.\\
Ät gyllene tosten, drick Foo Bars guld!

Uti vår Foo Bar finns Razor och bärs,\\
Drick Foo Bars guld\\
Men utav allt Zubr kärast oss är!\\
Drick Zubr och mera Zubr,\\
Och sen ännu mera Zubr,\\
Drick ljuvaste Zubrn, drick Foo Bars guld! \\

\newpage

%-------------------------------------------

\song{Trasan mår} \\       
\melo{Melodi: Helan går}
\author{Text: Hjärtat}

\songtext{} 
Trasan mår, \\
Inte jätte jätte bra,\\
Trasan mår,\\
Helt okej sådär.\\
Och den som inte ovven tar,\\
Den heller inte vraka får.\\
Trasan måååår,\\
Så jävla jävla bra!

\songinfo{* Skrevs av trasorna till insupet 2018.}\\

%-------------------------------------------

\song{Fest i Foo} \\       
\melo{Melodi: Rose Tattoo}
\author{Text: Lucifer, Hjärtat och Plinkplonk}

\songtext{Jag började som trasa}, liten som ett nyfött barn.  \\ 
Jag vaknade en onsdag, hade knappt ett öre kvar. \\ 
Hur fan ska jag kunna konsumera en öl för sjuttio spänn? \\ 
Studenter har det svårt med bara CSN.

Så jag föll ner på mina knän  \\ 
och bad högt till min gud. \\ 
"Åh Razor ge mig kraft och styrka jag vill ha det kul."  \\ 
Han svara: "Trasa är du korkad? \\ 
Klart finns en plats för dig  \\ 
en plats med mat, sprit, alkfritt för varje kille, varje tjej." 

Och vid 15:45 öppnade du dörren hem \\ 
marskalkar, vrak och trasor som aldrig kan få nog. \\
Åh Razor är min bästa vän  \\ 
jag vill aldrig åka hem igen, \\
Låt Zubrn flöda fritt för det är fest i Foo.

\leftrepeat  Det är fest i Foo.  \\ 
Det är fest i Foo. \\ 
Halsa whiskey, rom och gin när det är fest i Foo. \rightrepeat 

Det är fest i Foo. \\ 
Det är fest i Foo. \\ 
Vaknar upp uti en skog \\ 
efter en fest i Foo! 

\songinfo{*Framfördes insupet 2018 av Kelly-trasorna. Att vakna upp uti en skog refererar till en trasa som efter en pub mysteriskt vaknade dagen efter bland träden i Hallonbergen. Bör framträdas med banjo.} \\

\newpage

%-------------------------------------------

\song{KMS fel} \\       
\melo{Melodi: Bögarnas fel}
\author{Text: Frodo}

\songtext{Över hela denna NOD} - \textit{Oh yeah, oh yeah.}\\
Sker söndagstentor och Metod - \textit{Åh, ack och ve!} \\
Men ni som går med Razor har ändå ryggen fri \\
för roten till all världens ondska, det är inte ni.

Nej, det är KMS fel - \textit{KMS fel!} \\
KMS fel - \textit{KMS fel!} \\
I stadgan står det skrivet, \\
det är KMS fel. \\
Jag vet inte vilken paragraf, \\
kapitel eller del. \\
Men nånstans där i står det. \\
Det är KMS fel.

Mitt progtenta-resultat. \\
KMS fel. \\
Att CSN försvinner snart. \\
KMS fel. \\
KTH och Handels \\
och allt annat hemskt. \\
Att flaskan på Okanagan \\
blev så jävla sämst.
\newpage
Ja, det är KMS fel - \textit{KMS fel!} \\
KMS fel - \textit{KMS fel!} \\
Kista står i brand \\
och det är KMS fel. \\
På Kistan fick jag Innis \\
fast jag "nånting gott" beställt. \\
På nåt sätt kan jag känna, \\
det var KMS fel.

Mjölkdrink i ett mittenglas. \\
KMS fel. \\
Att Zubr-kannor går i kras. \\
KMS fel. \\
Att Qmisk nuförtiden, \\
serverar mindre bäsk. \\
Att dagens ungdom \\
numer bara köper alkoläsk.

Ja, det är KMS fel - \textit{KMS fel!}\\
KMS fel - \textit{KMS fel!}\\
Allt är upp och ner \\
Och det är KMS fel. \\
Min uppväxt gjort att \\
jag nog aldrig kan ha disciplin \\
men alla F som jag får \\ 
Det är KMS fel. 

\songinfo{* Skrevs katastrof-året 2020 när allt var KMS fel. \\ Det kursiva sjunges i falsett.}

\newpage 

%-------------------------------------------

\song{Fria Kista-drägg} \\       
\melo{Melodi: Yellow Submarine}
\author{Text: Lucifer}

\songtext{I Kista på 90-tal}  \\ 
fanns en skola med IT-drägg.  \\ 
Vår enda kår det var då SUS  \\ 
som ville styra och ställa krav.  \\ 
Vi bröt oss loss ur dess skit \\ 
och skapade oss en förening. \\ 
Våran enda oro nu \\ 
att ingen vet att vi finns här. \\
\leftrepeat Vi har blivit fria Kistadrägg. \\ 
Fria Kistadrägg. \\ 
Fria Kistadrägg. \rightrepeat

Även nu när vi lämnat \\ 
tror SUS ändå att vi är med. \\ 
Ger oss massa brevutskick \\ 
om att vara med i deras kår. \\ 
Kan ni snälla låt oss va'  \\ 
vi är ju inte er ex-flickvän. \\ 
Nu är nästan 30 år \\ 
sen vi gjorde vår DISK-exit. \\
\leftrepeat Vi har blivit fria Kistadrägg. \\ 
Fria Kistadrägg. \\ 
Fria Kistadrägg. \rightrepeat

\songinfo{* Skrevs till en sittning hos Juridiska Föreningen hösten 2019. Temat var Brexit.}

\newpage 

%-------------------------------------------

\song{Är det konstigt att man \\ längtar efter sprit?} \\       
\melo{Melodi: Är det konstigt att man längtar bort någon gång}
\author{Text: Jacke, Quiz och Yoshi}

\songtext{På DSV studerar man,} i vart fall officiellt \\
och längtar efter livet där \\
man har det bättre ställt. \\
För tillvaron är knaper  \\
och var dag är fylld med slit \\
Är det konstigt att man längtar efter sprit?

Säg är det konstigt att man längtar efter sprit? \\
När ens middagsmål är en nudelskål \\
och en kantstött sockerbit. \\
Alla tentorna man kuggar, \\
efter ändlösa nätter av flit. \\
Är det konstigt att man längtar efter sprit?

Så får man ett förvaltningsjobb i statsbyråkratin \\
så resistent mot kaffe \\
att man kissar koffein. \\
Kollegorna är torra \\
som en skiva masonit. \\
Är det konstigt att man längtar efter sprit?
\newpage
Säg är det konstigt att man längtar efter sprit? \\
Det är bra betalt men \\
är själsligt skralt \\
har man hamnat här med flit?
Börjar man ifrågasätta \\
varje vägval som ledde en hit? \\
Är det konstigt att man längtar efter sprit?

Där ute härjar pesten \\
och man fängslas i sitt hem. \\
Man lär sig för- och efternamn \\
på hela FHM. \\
När det har gått tolv månader \\
sen senaste visit. \\
Är det konstigt att man längtar efter sprit?

Säg är det konstigt att man längtar efter sprit? \\
Ingen dagsranson av desinfektion \\
bara vanlig akvavit. \\
Fullständigt isolerad \\
med ett skägg som en grå eremit. \\
Är det konstigt att man längtar efter sprit?

När hjärtat ligger krossat \\
eller huset brunnit ner. \\
När man har missat bussen \\
och det inte kommer fler. \\
När man behöver utstå \\
en spontanmormonvisit. \\
Är det konstigt att man längtar efter sprit?
\newpage
Säg är det konstigt att man längtar efter sprit? \\
Är man trött och less \\
och man känner mest att man \\
har åkt på en nit. \\
Man ser inget slut på ledan \\
och man söker en smula respit. \\
Är det klart som fan man längtar efter sprit!

\songinfo{*Framfördes som ett bidrag till Melodispexivalen 2021 - under pandemitiden}\\

%-------------------------------------------

\song{Far, jag kan inte få ner min flaska Stroh} \\       
\melo{Melodi: Far, jag kan inte få upp min kokosnöt}
\author{Text: Lucifer}

\songtext{}Far, jag kan inte få ner min flaska Stroh\\
Varje grogg jag prövat har vart pin \\
Med en lime och Sprite kved jag tills jag blev mör \\
Vinet det slank, från ölen blev pank, men flaskan den är full 

Far, jag kan inte få ner min flaska Stroh \\
Trots att hallonsoda är så gott\\
Varje vägg nu har spya här och var \\
Det är bara flaskan Stroh som är sig lik 
\newpage
Ja, det är bara flaskan Stroh som är sig lik \\
Alla på festen är nu raka som en spik \\
Säg mig, minns du badkar't, far? \\
Gör rent den är du rar \\
För det är bara toaletten som är sig ren 

Far, jag kan inte få ner min flaska Stroh \\
Nej, inte med ens yoghurt i mitt glas  \\
Ty hur jag nu svepte måste jag ha svept fel \\
För middan' kom upp, levern fick krupp, \\
Men flaskan den är full 

Far, jag kan inte få ner min flaska Stroh \\
Fast jag försökt knarka ner mig först \\
För näsblodet rann ner, och monstrena blev fler \\
Det är bara flaskan Stroh som var sig lik.

\songinfo{*Skrevs av Lucifer när han jobbade NärS tackfest och de hade tävling i snapsvisor. Han uppträdde inte men ville känna sig bättre än resten.}\\

%-------------------------------------------

\song{Jannes spritsång} \\       
\melo{Melodi: My Bonnie}
\author{Text: Janne Weijnitz}

\songtext{Min hjärna har slutat att funka},\\
min mage är full utav sår.\\
Min mor tror att det kan va' stressen\\
men jag vet varför det är så:\\
Spriten, spriten, så god, trevlig, rolig och underbar,\\
spriten, spriten, spriten har tagit mitt liv, hurra!

\newpage

%-------------------------------------------

\song{Visa till Foo Bar} \\       
\melo{Melodi: Vårvindar friska}

\songtext{} 
Nu ska ni tystna, nu alla lyssna,\\
alla svartklädda bland röda skal.\\
Spetsa nu sinnet, gräv fram ur minnet\\
bilden av KM:s forna lokal.\\
Spriten var billig, allting var bäst.\\
Ölen var kall, vi drack den på fest\\
Men dessa dagar flytt, vi beklagar,\\
Foo Bar, du finns ej mer.

Razor han gråter, känslor upplåter,\\
som ingen trodde fanns i hans själ\\
Hur ska vi orka tårarna torka,\\
när vi förstår hans sorger så väl?\\
”Gläds för det lilla” uppmanas kallt,\\
vi fick ett lekrum, men det var allt.\\
Hur ska vi glömma, sluta att drömma\\
om forna dagars glans?

Den nya tiden nu är framskriden,\\
framåt vi blickar, tar nya tag.\\
Somligt är dåligt, dumt, undermåligt,\\
men det blir bättre för varje dag.\\
Forum är hemmet som vi nu fått,\\
sluta att minnas dagar som gått.\\
Så fyll upp glasen och i extasen\\
finn er i ödets nyck.
\newpage
År räknas undan och med förundran\\
ser vi att KM friskt lever här.\\
Trots allt som hänt oss, gnistan har tänts hos\\
alla oss svarta med guldrevär.\\
Som om en vräkning skulle få oss\\
gå ner för räkning, hellre vi slåss!\\
För att upprätta hedern och sätta\\
fienderna på plats!

Banden av trohet, minnen om storhet,\\
binder oss samman starka som stål.\\
Razor hörs kalla, svartklädda alla\\
som ännu fler bataljer nu tål.\\
Foo Bar ska åter bli vår lokal!\\
Framåt nu genom tusende kval!\\
Skålen för Foo Bar lyftes om ni har\\
KM i hjärtat kvar.

Åter gryr dagen, timmen är slagen,\\
Foo Bar, vår moder, rätar sin rygg.\\
Razor hörs mysa, marskalkar rysa,\\
men nu finns nånstans att vara trygg.\\
Faran är över för denna gång,\\
ej mer behöver skrivas en sång,\\
om denna tiden, som är förliden\\
och blott fem verser lång.
\newpage
Gubben är sliten, trött, sammanbiten,\\
fru, barn och jobb har satt sina spår.\\
På dörren gläntar, vet vad som väntar,\\
Uppfylls av känslor just där vi står.\\
Dricker en öl och minns forna dar,\\
dricker en till och minns hur det var.\\
Sånger som ekar, kortspel och lekar,\\
Foo Bar, du står här kvar.

Skattjakt i Forum, sillunch i ölrum,\\
pulka i trappan och annat kul.\\
Punsch in absurdum, mögel i lekrum,\\
skärsår i händer och annat strul.\\
Svalda kapsyler, odygdig sång,\\
Byggare Bob en sommar så lång,\\
Mängder av minnen fyller upp sinnen,\\
Foo Bar - vad är på gång?

Åren har gått och i vått och torrt\\
har Foo Bar fått alla motstånd på fall.\\
Men åter viskar vindar som piskar\\
oron för vad som nu hända skall.\\
Nu ska du flytta ännu en gång,\\
åter vi måste väcka vår sång.\\
Slut er nu samman, på detta gamman,\\
stå upp mot detta tvång.
\newpage
Unga alerta styrker mitt hjärta,\\
visar på styrkan uti vår pakt.\\
Åter gryr ljuset, NOD heter huset,\\
Foo Bar finns åter i all sin prakt.\\
Här ska det ätas kräftor med dill,\\
pasta med guck, potatis med sill.\\
Här kan vi trivas, här kan vi kivas,\\
och mycket mer därtill.

Minnena bleknar, sinnena veknar,\\
vardagen tränger ohjälpligt på.\\
Tankarna vindlar, insikten svindlar,\\
hjärtat i bröstet börjar att slå.\\
Vrak faller undan och går i kvav,\\
nya marskalkar löser då av.\\
Kretsen är sluten, kedjan obruten.\\
Foo Bar, du blir min grav!


\songinfo{* De tre första verserna skrevs till DISK KM:s kräftskiva augusti 2001. Vers fyra och fem skrevs till internfesten september 2002 och den sjätte till internfesten februari 2004. Vers 7-9 är en bearbetning av det som Calle Almqvist och Anna Jenelius skrev till den sista internfesten i Forums Foo Bar juni 2014. Vers 10 och 11 framfördes av de närmast sörjande på insupet november 2014. Leve Foo Bar!}\\
