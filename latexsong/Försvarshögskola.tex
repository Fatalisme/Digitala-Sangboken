\song{Kungssången}
\author{Text: Carl Vilhelm August Strandberg}


\songtext{}Ur svenska hjärtans djup en gång,
en samfälld och en enkel sång,
som går till kungen fram!
Var honom trofast och hans ätt,
gör kronan på hans hjässa lätt,
och all din tro till honom sätt,
du folk av frejdad stam!

O konung! folkets majestät
är även ditt. Beskärma det
och värna det från fall!
Stå oss all världens härar mot,
vi blinka ej för deras hot,
vi lägga dem inför din fot,
en kunglig fotapall.

Men stundar ock vårt fall en dag,
från dina skuldror purpurn tag,
lyft av dig kronans tvång
och drag de kära färger på,
det gamla gula och det blå,
och med ett svärd i handen gå
till kamp och undergång!

Och grip vår sista fana du
och dristeliga för ännu
i döden dina män!
Ditt trogna folk med hjältemod
skall sömma av sitt bästa blod
en kunglig purpur varm och god,
och svepa dig i den.

Du himlens Herre! med oss var,
som förr du med oss varit har,
och liva på vår strand
det gamla lynnets art igen
hos sveakungen och hans män.
Och låt din ande vila än
utöver nordanland!





%-------------------------------------------------------------
\song{Flamma stolt}
\author{Text: Karl Gustav Ossiannilsson}

\songtext{}Flamma stolt mot dunkla skyar
lik en glimt av sommarns sol
över Sveriges skogar, berg och byar,
över vattnen av viol.
Du, som sjunger, när du bredes,
som vår gamla lyckas tolk:
"Solen lyser! Solen lyser!
Ingen vredes åska slog vårt tappra folk!"

Flamma högt, vår kärlekstecken,
värm oss, när det blåser kallt!
Stråla ur de blåa vecken,
kärlek, mera stark än allt!
Sveriges flagga, Sveriges ära,
fornklenod och framtidstolk,
Gud är med oss, Gud är med oss
Han skall bära stark vårt fria svenska folk.




%-------------------------------------------------------------
\song{Dåne liksom åskan}


\songtext{}Dåne liksom åskan bröder
högt vår fosterländska sång!
Pulsen brinner, hjärtat glöder!
Marsch framåt än en gång!
Sången ädla känslor föder.
hjärtats nyckel heter sång.

\leftrepeat Må vi då i toner svära
trohetseden hand i hand.
Liv och blod för Sveriges ära.
Hell, vårt dyra fosterland.
Liv och blod för Sveriges ära, svära
trogne bröder hand i hand. \rightrepeat




%-------------------------------------------------------------
\song{Sverige}
\melo{Melodi: }
\author{Text: }
\tags{}
\songinfo{here}

\songtext{}Sverige, Sverige, fosterland,
vår längtans bygd, vårt hem på jorden.
Nu spela källorna, där härar lysts av brand,
och dåd blev saga, men med hand vid hand
svär än ditt folk som förr de gamla trohetsorden.

Fall julesnö och susa djupa mo!
Brinn österstjärna genom junikvällen!
Sverige, moder! Bliv vår strid, vår ro,
du land där våra barn en gång få bo
och våra fäder sova under kyrkohällen!


%-------------------------------------------------------------
\song{God save the queen}
\tags{De internationella}


\songtext{} God save our gracious Queen,
Long live our noble Queen,
God save the Queen!
Send her victorious,
Happy and Glorious,
Long to reign over us;
God save the Queen!

O Lord our God arise,
Scatter her enemies
And make them fall;
Confound their politics,
Frustrate their knavish tricks,
On Thee our hopes we fix,
God, save us all!

Thy choicest gifts in store
On her be pleased to pour;
Long may she reign;
May she defend our laws,
And ever give us cause
To sing with heart and voice,
God save the Queen! 




%-------------------------------------------------------------
\song{Ja vi elsker dette landet}
\tags{De internationella}


\songtext{}Ja, vi elsker dette landet,
som det stiger frem,
furet, værbitt over vannet,
med de tusen hjem.
Elsker, elsker det og tenker
på vår far og mor
og den saganatt som senker
drømme på vår jord.
Og den saganatt som senker
senker drømme på vår jord.

Ja, vi elsker dette landet,
som det stiger frem,
furet, værbitt over vannet,
med de tusen hjem.
og som fedres kamp har hevet
det av nød til sejr,
også vi, når det blir krevet
for dets fred slår lejr,
også vi, når det blir krevet
for dets fred, dets fred slår lejr.




%-------------------------------------------------------------
\song{Vårt land (Maamme)}
\tags{De internationella}

\songtext{}Vårt land, vårt land, vårt fosterland,
ljud högt, o dyra ord!
Ej lyfts en höjd mot himlens rand,
ej sänks en dal, ej sköljs en strand,
mer älskad än vår bygd i nord,
än våra fäders jord.

Din blomning, sluten än i knopp, 
skall mogna ur sitt tvång; 
se, ur vår kärlek skall gå opp
ditt ljus, din glans, din fröjd, ditt hopp 
och högre klinga skall en gång 
vår fosterländska sång. 




%-------------------------------------------------------------
\song{Det er et yndigt land}
\tags{De internationella}


\songtext{}Der er et yndigt land,
det står med brede bøge
\leftrepeat nær salten østerstrand; \rightrepeat
det bugter sig i bakke, dal,
det hedder gamle Danmark,
\leftrepeat og det er Frejas sal. \rightrepeat

Det land endnu er skønt;
thi blå sig søen bælter,
\leftrepeat og løvet står så grønt, \rightrepeat
og ædle kvinder, skønne mø'r
og mænd og raske svende
\leftrepeat bebo de danskes øer. \rightrepeat




%-------------------------------------------------------------
\song{Norden}
\melo{Melodi: Högt över havet}
\songinfo{"Den där oljan de hittat i Norge låg ju där redan före 1905, så den lär väl egentligen  vara svensk?" -Tage Danielsson 1928-1985}

\songtext{}Titel: Finland
Melodi: Högt över havet

Finland är Finland och Finland är bra.
Dom har en pipeline med sprit från Moskva.
Bada Bastu, piska med ris,
hacka hål i is.
 
Danmark är Danmark och Danmark är bra.
Dom har en jungfru som sitter så bar.
Röde pölsor med Tuborg och lök,
vi köpte billig krök
 
Norge är Norge och Norge är bra.
Dom har den olja som vi vill ha.
Dyrt i baren ett jävla pris,
klubba säl med is.
 
Island är Island och Island är bra.
Kriser, vulkaner och hästar dom har.
Jag fiser i geisern vad var det jag sa, valspeck varje dag.

Ryssland är Ryssland och Ryssland är bra.
De tar en halvö som tillhör Ukraina.
Shotta vodka, och boffa lim,
invadera Krim!

Tyskland är Tyskland och Tyskland är bra.
de har en färja som alla vill ta.
Lasta bilen, lite för full
Fastnar i en tull.
 
Sverige är Sverige och Sverige är bäst.
Ingvar Kamprad han tjänar mest.
Ullared, Abba och Absolut,
Nu är visan slut.






%-------------------------------------------------------------
\song{Korvettvisan}
\melo{Melodi: I sommarenss soliga dagar }
\tags{De militära}

\songtext{}Från fänrikar till amiraler
hörs jubel och lyftas pokaler.
För flottans stoltaste fartyg
vi seglar på korvett, korvett, korvett.
Det bästa skepp, man nånsin' sett
sen amiralen var kadett.
En lillebror, hälften så stor
när jagarna till skroten for.
Men stoltast av alla på haven
vi seglar på korvett, korvett, korvett!




%-------------------------------------------------------------
\song{Kulan går}
\melo{Melodi: Helan går}
\tags{De militära, Helan går}

\songtext{}Kulan går
med avståndet rätt och vinden mätt
kulan går
om du bara riktar rätt
och den som inte kulan tar
får leva ännu några dar
kulan går
om du trycker av så nätt!




%-------------------------------------------------------------
\song{Lille Ivan}
\melo{Melodi: Bamsevisan}
\tags{De militära}

\songtext{}Lille Ivan sitter i sin ubåt,
sjunkbomb kommer, vatten strömmar in.
Ubåtsdelar ligger där på botten,
lille Ivan simmar aldrig mer
Blubb a blubb a blubb a blubb a blubb a blubb a 
blubb a blubb a blubb a blubb a blubb a blubb a
blubb a blubb a blubb

Lille Ivan sitter i sitt MiG-plan,
robot kommer, Ivan faller ner
flygplansdelar ligger där och brinner,
lille Ivan flyger aldrig mer.
Bom a bom a bom a bom a bom a bom a bom a
bom a bom a bom a bom a bom a bom a bom a
bom

Lille Ivan springer över schlätta
K-pist smattar, Ivan falller ner
Blod och slamsor ligger sprids utöver schlätta
lille Ivan springer aldrig mer

PANG



%-------------------------------------------------------------
\song{Flygarsupen}
\melo{Melodi: Flickan hon går i ringen}
\tags{De militära, Flygarvisor}

\songtext{}Vi flygare taga supar med fart och med kläm,
Vi slår dem i våra strupar en tre, fyra, fem.
Hå-hå, ja-ja, ja jäklar i dé.
Nu går den i magen i lodrät piké!

Vi flygare taga Halvan på vårt lilla sätt,
Vi störtar den ner i djupen uti vårt porträtt.
Hå-hå, ja-ja, vad livet är kort,
Den nubben skall säkert få fler i eskort!

Och Tersen den bliver trötter, att på bordet stå.
Och redan den börjat stiga mot himmelen blå.
Den skevar åt vänster och svänger en stund,
Gör sedan en looping och hamnar i mund.


%-------------------------------------------------------------
\song{fattig krigsman}
\melo{Melodi: Fattig bonddräng}
\tags{de militära}
\songinfo{"Ingen dikt har någonsinn skrivits av en vattendrickare" -Horatio 65-8 f kr}

\songtext{}Jag är en fattig krigsman men jag lever äbå.
Dagar går och kommer medan jag spårar på.
Marschar, går och skjuter, plockar hylsor och bär.
Går bak mina hundar observerar och svär.

Jag är en fattig krigsman och jag hyllar mitt snus.
Och när löning kommer vill jag ta mig ett rus.
Sen, när jag blivit livad vill jag tampas och slåss.
Vila hos en flicka vill jag också förstås.

Sen, Så kommer månda'n och då vill våran chef,
Att jag ska till gymmet, men då sover jag helst.
Chefen kan väl sova hela måndagen men,
För en fattig krigsman börjar knoget igen.

Så gårhela veckan, alla dagar och år.
Jag går med min AK, och jag skjuter och slår.
Jag kör mina hundar och jag vaktar mitt slott.
Spårar, gnor och trälar, och till sist går jag bort.



%-------------------------------------------------------------
\song{Brev från flyget}
\melo{Melodi: Brev från kolonien}

\songtext{}MiG:en startar, SIS:en vakar.
Blue Shark gapar, JAS:en jagar,
Varje päsk och varje annan storhelg
ryssen älskar vara FISK:ens lilla revelj.

Ut pả havet, seglar flottan,
dricker punsch och spyr i pottanm
skepp pả grund och lossa tătar,
flottan lyckas inte ens jaga ubatar.

Infanteriet, de är klena
Pansar fastnat, Art är sena.
Vilken tur att vi har Gripen
den hjälper gärna till och glider nedför strip(p)en.

Ja, tänk vad flyget häller ställning
när flottan snurrar, missar minfällning 
och stackars Jägarn som sitter och kikar
de käkar VA-mat och bajsar tretumsspikar.

Ja, tack och hej fràn stolta flyget.
Vi vet ni ger oss, högsta betyget.
Men vi trivs väst i innebandysalen
Eller helst i fikarummet och hangaren.

%-------------------------------------------
\song{Ode till flygbensinet}
\melo{Melodi: helan går}
\tags{Flygarvisor, Helan går, De militära}

\songtext{}Flygbensin är drycken för vart fyllesvin
Flygbensin är rått som terpentin
Det piggar upp din trötta kropp
och hettar upp ditt blodomlopp
Flygbensin ...
är Mannfreds medicin

%-------------------------------------------
\song{Pang Pang Pangsarvagn}
\melo{Melodi: Bä Bä vita lamm}
\tags{De militära}

\songtext{}Pang Pang Pangsarvagn
har du något krut
Nej, nej inte jag
det har tagit slut
Ska vi ta en snaps 
eller kanske två
men ej vodka 
Ryssen kommer då

%-------------------------------------------
\song{Molnen}
\melo{Melodi: Nu är det jul}
\tags{Flygarvisor, De militära}

\songtext{}Nu tar vi fart, flyget så klart
Vädret är toppen.
Hopp tralala
Tornet sa nu, startbana sju, startbana sju.

Vi flyger så högt vi kan i luften.
Vi flyger så högt vi kan i luften.
Bara högre, ännu högre, ännu högre i molnen.

Jag ser ingenting, moln runtomkring.
Motorn har stannat.
Hopp tralala
Planet i stall, det här var ju ball, det här var ju ball.

Vi flyger så högt vi kan i luften.
Vi flyger så högt vi kan i luften.
Bara högre, ännu högre, ännu högre i molnen.

Vi störtar så klart, hoppa i fart.
Planet det krachat.
Hopp tralala
Ingen skärm ut , nu är det slut, nu är det slut.

%-------------------------------------------
\song{Nationalsnapsen}
\melo{Melodi: Du gamla du fria}

\songtext{}Du gamla, du fina, du storsvenska snaps,
snart randas ditt glädjedöda öde.
Nog minns vi med saknad den tid då krånglet slapps,
\leftrepeat Ditt låga pris och dina mängders flöde \rightrepeat

Jag tronar på guld från en lånad bankir,
när chartrat mitt plan flyr Sveriges torka.
Jag minns hur det var och jag anar hur det blir.
\leftrepeat Ja, jag vill leva loppan på Mallorca \rightrepeat

%-------------------------------------------
\song{Älgen Hans}
\melo{Melodi: Helan går}
\tags{Helan går}

\songtext{} Älgen Hans, han äter både sten och grus
Älgen Hans, han välter stora hus
Älgen, älgen, älgen, älgen
Älgen, älgen, älgen, älgen
Älgen Hans......
...han har så stora horn!

%-------------------------------------------
\song{Öl, vin, sprit}
\melo{Melodi: Jenka}

\songtext{} Öl, vin sprit och gammal finkel,
har fått mig att se i vinkel.
Darför hamnar all min mat 
ej i munnen utan i min
hörapparat.

%-------------------------------------------
\song{Mitt lilla lån}
\melo{Melodi: Hej Tomtegubbar}

\songtext{} \leftrepeat Mitt lilla lån det räcker inte till,
det går till öl och brännvin \rightrepeat
Till öl och brännvin går det åt
och till små böcker emellanåt.
Mitt lilla lån det räcker inte till, 
det går åt till öl och brännvin.

%-------------------------------------------
\song{Till mässen ränner jag}
\melo{Melodi: Snickerboa}
\songinfo{}Till FHS mäss ränner jag 
när plugget ej blitt gjort.
När ångesten tar övertag,
ska pengar ryka fort.
HSU-arn sade goddag, 
hur full tänkte du bli idag?
Passerkort tack, ditt fula fuck,
Ja, annars kallar jag på vakt.

Till Sverigesalen ränner jag,
för jag är jättesen.
Igår var mässen full med folk, 
Jag spydde på Bydén.
In i salen stormar jag
Hur bakfull lär jag bli idag?
Ogonkontakt, nu är jag fucked
Lärar'n kallat på en vakt

%---------------------------------------------
\song{Kom och häng i mässen}
\melodi{Melodi: Här kommer Pippi Långstrump}

\songtext{}Kom och häng i mässen
tjolahopp tjolahej, tjolahoppsansa
kom och häng i mässen
tjolahej tjolahoppsansa

Har du sett mitt schema
allt som ska göras denna vecka?
Har du sett mitt schema
för det har faktiskt jag
Tiden är dyrbar
har både disk och plugg och hyra
Men mage fylld med dricka
är ju också bra att ha

Sả kom och häng i mässen
tjolahopp tjolahej, tjolahoppsansa
Kom och glöm i mässen
tjolahej tjolahoppsansa.