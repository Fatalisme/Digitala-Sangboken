







\song{A-Sektionen}
\melo{Melodi: Rule, Britannia}
\tags{KTH sektioner}

\songtext{}A-sektionen, den skiner så som solen,
A-sek är bäst på fest, A-sek är bäst


%-------------------------------------------------------------

\song{D-Sektionen}
\melo{Melodi: Who´s that lying on the runway}
\tags{KTH sektioner}

\songtext{Vem e störst och bäst i Skåne?
Vem e kung på LTH, på LTH?
Ja det e Data Infocom som hela festen drar igång
Tacka gud för Data Infocom!}


%-------------------------------------------------------------


\song{E-Sektionen}
\melo{}
\author{}
\tags{KTH sektioner}

\songtext{Everywhere we go (Everywhere we go)
People wanna know (People wanna know)
Who we Are (Who we Are)
So we tell them (So we tell them)
We are the E-sek (We are the E-sek)
Mighty mighty E-sek (Mighty mighty E-sek)
Oh ah o E-sek!}


%-------------------------------------------------------------


\song{F-Sektionen}
\melo{}
\author{}
\tags{KTH sektioner}

\songtext{Vi går på F-sek, F-sek vi går på F-sek
Derivera Sin(x) så får du Cos(x)
Cos(x) Cos(x) vi vill ha Cos(x)
Vill vi? Neeej!
Vi går på F-sek, F-sek vi går på F-sek
Derivera Sin(x) så får du Cos(x)
Cos(x) Cos(x) vi vill ha Cos(x)
Vill vi? Neeej, vi vill ha ÄLGSEX!}


%-------------------------------------------------------------


\song{ING-Sektionen}
\melo{Melodi: Robin Hood Rooster song}
\author{}
\tags{KTH sektioner}

\songtext{För vi är ING från sundets pärla,
och det är fest idag igen,
o vi ska supa hela natten lång,
o sjunga den sång!}



%-------------------------------------------------------------


\song{K-Sektionen}
\melo{Melodi: What shall we do with a drunken sailor?}
\author{}
\tags{KTH sektioner}

\songtext{Vilka har den starka rØsten
Vilka har de stØrsta brØsten
Vilka ger er alltid trØsten
När ert lag förlorar
KK
Vi alla gillar
KK
Har faktiskt killar
KK
Som gärna pillar
Med sin labbutrustning}


%-------------------------------------------------------------


\song{M-Sektionen}
\melo{Melodi: When the Saint Go Marching In}
\author{}
\tags{KTH sektioner}

\songtext{Vår färg är röd,
vår färg är fin,
för det är vi som går maskin.
Och vi har kommit för att dricka alkohol
För det är vi som går maskin!}


%-------------------------------------------------------------


\song{V-Sektionen}
\melo{Melodi: When the Saint Go Marching In}
\author{}
\tags{KTH sektioner}

\songtext{Åh när dom blå (Åh när dom blå)
Åh när dom blå (Åh när dom blå)
Åh när dom blå marscherar in
Då blir det fest, då blir det jubel
Åh när dom blå marscherar in}


%-------------------------------------------------------------


\song{W-Sektionen}
\melo{}
\author{}
\tags{KTH sektioner}

\songtext{Aha, ekosång! Alla eko:sar kom igång! (x5)}


%-------------------------------------------------------------


\song{Konglig Fysiks Paradhymn}
\melo{Melodi: Katyuscha}
\author{Dum-Dum 1977}
\tags{Fysiker, FKM}

\songtext{}Här på festen stiger åter glammet.
Sången börjar, tentan bortglömd är.
Lyss min strupe Du plågas utav dammet,
frukta ej ty hjälpen är just här
ibland oss.
Höj pokalen, dess flöde känns som sammet.
Drick till det som Bacchi vapen lär.

Känn, Quristina, känn hur blodet hettar.
Du, o Osquar, orsak till det är.
Timmar skrider och dygdens bojor lättar.
Fest och glädje kärleksflamman när
men minns att
blott ej synen en hungrig kärlek mättar.
Drick till det som Venus' vapen lär.

FYSIKER, gasqueropen de har skallat,
likt musik från någon högre sfär.
Tentans piska för länge har oss vallat.
Trotsa den och studiernas misär,
med lärdom
från de makter som ytterst har oss kallat
Bacchus, Venus värdar hos oss är.
Och vänner,
Bacchi nektar ej Venus' flamma släcker.
SKÅL för det Fysiks skyddsgudar lär.

\songinfo{Under sista versen står F-teknologer, men inga andra, upp. ''Osquarulda'' kan bytas mot ''Osquar'', ''Quristina'' eller ''Osqulda'' efter behag.}


%-------------------------------------------------------------


\song{Årskursernas hederssång}
\melo{}
\author{}
\tags{Fysiker, FKM}

\songtext{}Alla: För det var i vår ungdoms fagraste vår,
vi drack varandra till och vi sade ''gutår''.
Och alla så dricka vi nu Foo till,
Foo:Och Foo säger inte nej därtill...

Vid Foo insättes n-m,
där n=aktuellt årtal, m=0, 1, 2...

23 Flipp/Flopp
22 Flott
21 Fnatt
20 Fotnot
19 Fasett



%-------------------------------------------------------------


\song{De Brevitate Vitae}
\melo{}
\author{}
\tags{Fysiker, FKM}

\songtext{||: Gaudeamus igitur, iuvenes dum sumus! :||
Post icundam iuventutem, post molestam senectutem
||: nos habebit humus. :||

||: Ubi sunt, qui ante nos in mundo fuere? :||
Vadite ad superos, transite ad inferos,
||: ubi iam fuere! :||

||: Vita nostra brevis est, brevi finietur, :||
venit mors velociter, rapit nos atrociter,
||: nemini parcetur. :||

||: Vivat academia, vivant professores! :||
Vivat membra quaelibet, vivant membran quaelibet
||: semper sint in flore! :||

||: Vivant omnes virgines, graciles formosae! :||
Vivant et mulieres, tenerae,
||: amabiles bonae laboriosae! :||

||: Vivat et respublica et qui illam regit! :||
Vivat nostra civitas, Maecenatum caritas,
||: quae nos hic protegit! :||

||: Pereat tristitia, pereant osores! :||
Pereat diabolus, quivis antiburschius,
||: atque irrisores! :||}

\songinfo{Biskop Stada 1267, C.W. Kindleben 1781}


%-------------------------------------------------------------


\song{Överföhssång}
\melo{Melodi: Kungssången}
\author{}
\tags{Fysiker, FKM}

\songtext{Ur teknologens djup en gång
en samfälld och en enkel sång
som går till Överföhs fram

Var honom trofast och hans ätt
Gör piskan i hans näve lätt
Och all din tro till honom sätt
Du schlemm av ingen rang}

\songinfo{Skriven till Ettans Fest 1990}


%-------------------------------------------------------------


\song{När Fumla blev fem}
\melo{Melodi: När vi två blir en (Gyllene Tider)}
\author{}
\tags{Fysiker, FKM}

\songtext{Ettan:
Det verkade lätt, att gå på KTH.
Jag var den med alla rätt, som ingen rådde på,
men det é så svårt å hinna med allt,
jag menar, plugga non-stop é inte så ballt.
Jag trodde jag var smart, men teknisk fysik, va inte självklart.

Alla:
Så vi dränker vår sorg med en fest.
Hör nu glasen slå! (klinga glasen)
Ty vårt valmanifest\
är att shotta lite stroh. (ROM!)
Så lyft era glas, nu är det kalas
för termon ja ja, den gick väl bra.
Så vi dränker vår sorg med en fest,\
nu börjar år 2.

Tvåan:\
Det verkade fett, att gå i årskurs två.
Om vi grejade år ett, så borde detta gå,
men att gasqua i kons är lite för kul,
så paniken slog till, lagom till jul.
Vi tar bäsken med en min, och tentan får jag, klara nästa termin.

Alla:
Så vi dränker vår sorg med en fest.
Hör nu glasen slå! (klinga glasen)
Ty vårt valmanifest
är att shotta lite stroh. (ROM!)
Så lyft era glas, nu är det kalas
för diffen ja ja, den gick väl bra.
Så vi dränker vår sorg med en fest,
nu börjar år 3.

Trean:
Det verkade hett, att ta ett år med sol,
få leva mer jet set, å vara lite cool.
Men det é svårt å plocka poäng
när man hellre festar loss i en simbassäng
sen vare plötsligt vår, med fasta och kex, jobbet består.

Alla:
Så vi dränker vår sorg med en fest.
Hör nu glasen slå! (klinga glasen)
Ty vårt valmanifest
är att shotta lite stroh. (ROM!)
Så lyft era glas, nu är det kalas
för kvanten ja ja, den gick väl bra.
Så vi dränker vår sorg med en fest,
nu kommer vi hem.

Fyran:
Det verkade rätt, att grunda med komplex.
Ja nu blir mastern lätt, men vårat hopp det släcks,
för det é så svårt å göra inlupp
när man aldrig lärt sig jobba i grupp.
Det blir mest en massa prat, å då blir det svårt, inga toppresultat.

Alla:
Så vi dränker vår sorg med en fest.
Hör nu glasen slå! (klinga glasen)
Ty vårt valmanifest
är att shotta lite stroh. (ROM!)
Så lyft era glas, nu är det kalas
för mastern ja ja, den går väl bra.
Så vi dränker vår sorg med en fest,
nu börjar år 5.

Femman:
Hur har detta skett, alla åren rullar på.
Slipat CV, skötsamt sätt, så ska jag jobbet nå.
För nu är det dags, att tjäna sin deg
eller bli doktorand för å slippa kneg.
Måste lämna inom kort, så vi filar på vår exjobbsrapport.

Alla:
Så vi dränker vår sorg med en fest.
Hör nu glasen slå! (klinga glasen)
Ty vårt valmanifest
är att shotta lite stroh. (ROM!)
Så lyft era glas, nu är det kalas
för examen ja ja, den blir nog bra.
Så vi dränker vår sorg med en fest,
å nu går vi ut.

Seniorer:
Det var ju så lätt, att gå på KTH,
Har egen bostadsrätt, har nått en ny nivå.
De som pluggar nu, de har ingen koll,
jag menar dagens ungdom, de fattar noll.
Men en sak lever kvar, ingen har glömt, hur bra vi var.

Alla:
När vi dränkte vår sorg med en fest,
å hörde glasen slå! (klinga glasen)
Ty vårt valmanifest
var att shotta MASSA stroh! (ROM!)
Så lyft era glas, nu är det kalas
för livet ja ja, vi har det bra.
Fast vi saknar att vara på fest
för vi har gått ut.}

\songinfo{F-12 Fumla; Jubilariegyckel, Ettans fest 2017}


%-------------------------------------------------------------


