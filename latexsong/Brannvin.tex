
\headertext{brännvin}
\begin{centering}
\Large \textbf{Suparnas ordning} \index{Suparnas ordning} \\
\vspace{0.5cm}
\small
Helan

Halvan

Tersen

Kvarten

Kvinten

Rivan\\
Septen\\
Rafflan\\
Rännan\\
Smuttan\\
Smuttans unge\\
Femton droppar\\
Lilla Manasse\\
Lilla Manasses broder\\
Klämtaren\\
Kreaturens uppståndelse\\
Den bleka dödens dryck\\
Razors sängfösare\\
\end{centering}
\clearpage
%---------------------------------------------------------
\song{Kort finsk supvisa}
\vspace{0.15cm}
\melo{} \\
\songtext{}\textit{17 sekunder tystnad.}\\
Nu! \\

%-------------------------------------------
\song{Lång finsk supvisa}
\vspace{0.15cm}
\melo{} \\
\songtext{}Int' nu... \\ 
\textit{17 sekunder tystnad.}\\
 ...men nu! \\

%-------------------------------------------
\song{Helan går} 
\vspace{0.15cm}
\melochtext{} \\
\songtext{}Helan går,\\
sjung hopp, faderallanlallanlej.\\
Helan går,\\
sjung hopp, faderallanlej.\\
Och den som inte helan tar\\
han heller inte halvan får\\
Helan går!\\
Sjung hopp, faderallanlej! \\

\newpage

%-------------------------------------------
\song{Alla tallarna} \\       
\author{Text: Lars T. Johansson och Ehrling Eliasson}

\songtext{}
Alla tallarna, alla tallarna, alla stora, alla små!\\
Alla tallarna, alla tallarna, ska vi koka 'rännvin på!\\
Alla tallarna, alla tallarna, ifrån roten till dess topp,\\
Alla tallarna, alla tallarna, ska vi ta och 'ricka opp!\\
Skål! 

\songinfo{* Ska sjungas i falsett} \\

%-------------------------------------------
\song{Måsen} \\       
\melo{Melodi: När månen vandrar på fästet blå}

\songtext{Det satt en mås på en klyvarbom},\\
och tom i krävan var kräket.\\
Och tungan lådde vid skepparns gom\\
där han satt uti bleket.\\
Jag vill ha sill hördes måsen rope\\
och skepparn svarte: jag vill ha OP\\
om blott jag får, om blott jag får.

Nu lyfter måsen från klyvarbom\\
och vinden spelar i tågen.\\
OP:n svalkat har skepparns gom,\\
jag önskar blott att jag såg en.\\
Så nöjd och lycklig den arme saten,\\
han sätter storsegel den krabaten,\\
till sjöss han far och halvan tar!
\newpage
Nu månen vandrar sin tysta ban\\
och tittar in i kajutan.\\
Då tänker jag att på ljusa da'n\\
då kan jag klara mig utan.\\
Då kan jag klara mig utan måne,\\
men utan Renat och utan Skåne,\\
det vete fan, det vete fan.

Den mås som satt på en klyvarbom,\\
den är nu död och begraven,\\
och skepparn som drack en flaska rom,\\
han har nu drunknat i haven.\\
Så kan det gå om man fått för mycket,\\
om man för brännvin har fattat tycke.\\
Vi som har kvar, vi resten tar. \\

%-------------------------------------------
\song{JASen} \\       
\melo{Melodi: När månen vandrar på fästet blå}

\songtext{}Det flög en JAS över västerbron \\
men styrsystemet var trasigt.\\
Piloten ut sköt sig med kanon\\
för planet vingla' så knasigt.\\
Han ville uppåt, han ville mer\\
men planet svarte: ”Jag vill ju ner\\
mot alla hjon, på Västerbron.” 

\songinfo{* Skriven av dadderiet vid Datasektionen, KTH, i samband med n0llningen 1993.} 

\newpage

%-------------------------------------------
\song{Balrogen} \\       
\melo{Melodi: När månen vandrar på fästet blå}

\songtext{Det satt en trollkarl på västerbron}\\
För han trodde den magisk\\
Men bron den visa sig va mundan\\
Och trollkarln’ framstod som tragisk.\\ 
Det kom en balrog, den ville förbi\\
Men trollkarln’ svarte "det skiter jag i;\\ 
Ditt jävla as! You shall not pass!" \\

%-------------------------------------------

\song{Musen} \\       
\melo{Melodi: När månen vandrar på fästet blå}

\songtext{}Det satt en mus i en hushållsost\\
och åt och åt utan måtta\\ 
tills osten blev till en mushåls-ost\\ 
och han en klotformad råtta.\\ 
”Så bra”, sa musen ”att va en fettboll\\ 
nu kan jag rulla med hast åt rätt håll:\\ 
Ostindien, Ostindien.” \\

\newpage 

%-------------------------------------------

\song{Moosen} \\       
\melo{Melodi: När månen vandrar på fästet blå}

\songtext{}Det satt en älg i en klyvartopp,\\
förklädd i älgjaktens månad.\\
Han var befjädrad till horn och kropp\\ 
Ja, skepparn blev rätt förvånad.\\
”Jag är en mås, goa skepparn”, ljög den\\ 
förklädda älgen, därefter flög den.\\ 
Mjukt föll den sen, på skepparen. \\

%-------------------------------------------

\song{När nubben blänker} \\       
\melo{Melodi: När månen vandrar på fästet blå}

\songtext{}När nubben blänker i immigt glas \\
som hoppets strålande stjärna, \\
då är det avsett att det ska tas \\
förutan fruktan och gärna. \\
Så klang och klingom, så tar vi supen, \\
den läskar härligt den torra strupen. \\
Ja, skål gutår, ja, skål gutår! \\


\newpage

%-------------------------------------------

\song{Mesen} \\       
\melo{Melodi: När månen vandrar på fästet blå}

\songtext{}Det satt en mes i en klyvarmast,\\
där sågs hen ragla och svaja.\\
För trots att frön var hens enda last,\\
var hen nu full som en kaja.\\
"Vad har du gjort!" hördes skepparn stöna\\
och mesen svarte: "Jag rökte fröna!\\
I egen holk, i egen holk" \\

%-------------------------------------------

\song{Boeing} \\       
\melo{Melodi: När månen vandrar på fästet blå}
\author{Text: Jacke}

\songtext{}Det flög en Boeing mot Vietnam \\
men något hände på vägen \\
För planet girade och försvann \\
Var någon kaparbenägen? \\
Kom nån och strippade för piloten? \\
Söp han på jobbet, den idioten!? \\
Ett haveri - nu super vi! \\

\newpage

%-------------------------------------------

\song{Capsen} \\       
\melo{Melodi: När månen vandrar på fästet blå}

\songtext{}Det var en caps utav bästa sort\\
och trasan fick sista platsen.\\
Men hon capsa på tok för kort,\\
slaska' ut allt sitt vatten.\\
Mot alla vraken det blev duell,\\
för just som vrak vill man vara snäll.\\
Den trasan dog, den trasan dog.

\songinfo{* 14/9 2004 arrade vraken en sittning för att trasorna skulle få uppleva en bra sådan. Flera sammanhängande spex var upplagda som en mordgåta av Cluedo-typ och i början av sittningen dog därför ”Trasa Grå” oväntat. Dödsfallet utreddes noggrant och trots flera bisarra misstänkta visade sig allt bero på en svald kapsyl.}\\

%-------------------------------------------

\song{Örnen} \\       
\melo{Melodi: När månen vandrar på fästet blå}
\author{Text: Doomy}

\songtext{}Det satt en örn uti Foo Bar\\
Och han drog sig i lemmen\\ 
Men han slarvig med gylfen var\\ 
Så hans kuk kom i klämmen\\ 
Vad gör man då, när ens pung fått punka?\\ 
Ja, våran Örn, han fortsatte runka\\ 
Och Örn han kom, så småningom. \\


\newpage

%-------------------------------------------

\song{Måsen (kort)} 
\vspace{0.15cm}
\melochtext{} \\
\songtext{}Måsen, måsen\\
Drick, drick, drick! \\

%-------------------------------------------

\song{Mera brännvin
} \\       
\melo{Melodi: Internationalen}

\songtext{Nu är det dags att taga supen},\\
den stärker varje svag fysik.\\
Den rinner ner igenom strupen,\\
river gott som en tolvtums spik.

Den är vårt hopp mot gula faran,\\
vår tröst vid varje bleklagd sorg.\\
Den stärker oss mot mask i magen,\\
starkare än Sveaborg.

Mera brännvin i glasen,\\
mera glas på vårt bord,\\
mera bord på kalasen,\\
mer kalas på vår jord.

Mera jordar kring månen,\\
mera månar kring mars,\\
mera marscher till Skåne,\\
mera Skåne gud bevars!\\

\newpage 

%-------------------------------------------

\song{Livet är härligt} \\       
\melo{Melodi: Röda kavalleriet}

\songtext{}
Livet är härligt!\\
Tavaritj, vårt liv är härligt!\\
Vi alla våra små bekymmer glömmer\\
när vi har fått en tår på tanden, skål!

Ta dig en vodka!\\
Tavaritj, en liten vodka!\\
Glasen i botten vi tillsammans tömmer,\\
det kommer mera efter hand. En skål!

\songinfo{* Från Chalmersspexet "Katarina II" 1959.} 

\songtext{} Fingret i halsen!\\
Tavaritj, ett stick i halsen!\\
Magen på golvet vi tillsammans tömmer.\\
det kommer mera efter handen - skål!

\songinfo{* Tillägg från den finlandssvenska studentföreningen Spektrum} \\

\newpage

%-------------------------------------------

\song{Imbelupet} \\       
\melo{Melodi: Kors på Idas grav}

\songtext{}
Imbelupet glaset står på bräcklig fot.\\
Tomma pilsnerflaskor luta sig därmot.\\
Men där nere, miserere,\\
uti magens dunkla djup,\\
sitter djävulen och väntar på en sup!

Impelupet glaset...\\
...uti magens mörka valv,\\
sitter djävulen och väntar på en halv!

Impelupet glaset...\\
...uti magen härs och tvärs,\\
sitter djävulen och väntar på en ters!

Impelupet glaset...\\
...uti magen tom och svart,\\
sitter djävulen och väntar på en kvart!

Impelupet glaset...\\
...uti magens labyrint,\\
sitter djävulen och väntar på en kvint!

Impelupet glaset...\\
...uti magens slingerväxt,\\
sitter djävulen och väntar på en sext!

Impelupet glaset...\\
...uti magen halvuppknäppt,\\
sitter djävulen och väntar på en sept!

Impelupet glaset...\\
...uti magen an och av,\\
vankar själve fan och väser på oktav!

Impelupet glaset...\\
...sitter allas våran far,\\
det är Fan och han vill ha det som är kvar! \\

%-------------------------------------------

\song{Vikingen} \\       
\melo{Melodi: When Johnny comes marching home
}

\songtext{En viking älskar livets vand},\\
hurra, hurra!\\
Den hastigt i mitt svalg försvann,\\
hurra, hurra!\\
Till kalv, till oxe, till fisk, till fläsk,\\
när kärringar bara dricker läsk.\\
Ja, då vill alla vikingar ha en bäsk.

När bäsken småningom är slut,\\
tragik, tragik\\
Då bäres varje viking ut\\
som lik, sig lik.\\
Och se'n, om vi vaknar, vi sjunger en bit,\\
se'n korkar vi upp Skånes Aquavit.\\
Skål för alla vikingar som kom hit!\\

\newpage
%-------------------------------------------

\song{Sup dig snabb} \\       
\melo{Melodi: Drunken sailor}

\songtext{}
Om du känner dig fet och sliski',\\
fyll ditt glas, prova på vår whisky,\\
då blir du både glad och frisk i, \\
kropp såväl som själen

Hej skål jag svingar supen\\
Hej skål jag bringar supen\\
Hej skål jag vingar supen\\
svävar genom strupen.\\

%-------------------------------------------
\song{En gång i måna'n} \\       
\melo{Melodi: Mors lille Olle
}

\songtext{}
En gång i måna'n är månen full,\\
aldrig jag sett honom ramla omkull.\\
Stum av beundran hur mycket han tål,\\
höjer jag glaset och säger nu skål.

Höjer nu glasen och dricker ur.\\
Nu kära bröder, står halvan i tur.\\
Nubben den giver oss ny energi,\\
säkert den minskar vårt livs entropi.\\

\newpage

%-------------------------------------------

\song{Än en gång däran} \\       
\melo{Text och musik: Evert Taube}

\songtext{}
Än en gång däran, bröder! Än en gång däran!\\
Följom den urgamla seden!\\
Intill sista man, bröder, intill sista man,\\
trotsa vi hatet och vreden!\\
Blankare vapen sågs aldrig i en här,\\
än dessa glasen, kamrater: \textbf{I gevär!}\\
Än en gång däran, bröder! Än en gång däran!\\
Svenska hjärtans djup - här är din sup!

Livet är så kort, bröder! Livet är så kort!\\
Lek det ej bort, nej var redo!\\
Kämpa mot allt torrt, bröder! Kämpa mot allt torrt!\\
Tänk på de gamle som skredo\\
fram utan tvekan i floder av champagne,\\
styrkta från början av brännvin från vårt land!\\
Kämpa mot allt torrt, bröder! Kämpa mot allt torrt!\\
Svenska hjärtans djup - här är din sup!\\
Skrevs i början av 1930-talet.\\

\newpage 

%-------------------------------------------

\song{Törsten rasar} \\       
\melo{Melodi: Längtan till landet}

\songtext{}
Törsten rasar uti våra strupar.\\
Tungan hänger torr och styv och stel.\\
Men snart vankas stora långa supar\\
var och en får sin beskärda del.\\
Snapsen kommer, den vi vilja tömma,\\
denna nektar likt Olympens saft.\\
Kommer oss att våra sorger glömma\\
snapsen skänker hälsa, liv och kraft.

Fordom odlade man vindruvsranka,\\
av vars frukt man gjorde ädelt vin.\\
Nu man pressar saften ur en planka,\\
doftande av äkta terpentin.\\
Höj nu bägaren, o bröder och systrar,\\
låt den svenska skogen rinna kall,\\
ned i strupen, och om du är dyster:\\
Låt oss dricka upp en liten tall!

Helan tänder helig eld i själen\\
halvan rosar livet som en sky.\\
Tersen känns från hjässan ner i hälen\\
kvarten gör en som en mänska ny.\\
Låt oss skåla med varann go' vänner,\\
skål för våran levnads glada hopp.\\
Törstens eld på nytt i strupen bränner.\\
Leva livet! Skål och botten opp!

\newpage 

%-------------------------------------------

\song{Denna thaft} \\       
\melo{Melodi: Helan går}

\songtext{}
Denna thaft, den bästa thaft thythtemet haft. \\
Denna thaft är den bätha thaft dom haft. \\
Och den thom inte har nån kraft \\
han dricka thkall av denna thaft. \\
denna thaft, till landth, till thjöth, till havth! \\

%-------------------------------------------

\song{Fillibrännvinbom} \\       
\melo{Melodi: Ritsch, ratsch, fillibom}

\songtext{}Brännvin fillibom-bom-bom \\
är en härlig dryck för en törstig gom. \\
Brännvin fillibom-bom-bom \\
är mitt livs potatis-jom. \\
Ett litet barn vid flaskan redan blivit van. \\
Det sitter i, till dess vi gamla bli. \\
Så följ de gamla lagarna, \\
drick något starkt om dagarna, \\
för det är bra för magarna \\
och stärker vår aptit. \\

%-------------------------------------------

\song{Vad blåser det för vind idag?}    
\vspace{0.15cm}
\melochtext{} \\
\songtext{}Vad blåser det för vind idag? \\
- Brännvind! 

\newpage

%-------------------------------------------

\song{Sjömannens visa}
\vspace{0.15cm}
\melochtext{} \\
\songtext{}Åh, boj! \\

%-------------------------------------------

\song{Cykelhandlarens visa}     
\vspace{0.15cm}
\melochtext{} \\
\songtext{}Åh, hoj!\\

%-------------------------------------------

\song{Regissörens skål} 
\vspace{0.15cm}
\melochtext{}  \\
\songtext{}Tystnad... \\
Tagning! \\

%-------------------------------------------

\song{Farväl, farväl} \\       
\melo{Melodi: En sjöman älskar havets våg} 

\songtext{} Farväl, farväl, förtjusande sup, \\
men kom inte upp igen! \\

%-------------------------------------------

\song{Hyfsvisa} 
\vspace{0.15cm}
\melo{} \\
\songtext{Kors i allsin dar!}\\
Har du brännvin kvar? \\
Är du sparsam eller snål? \\
Skål! 

\songinfo{* Sparsamma sjunger "Snål!"} 
\newpage
%-------------------------------------------

\song{Jag äter inte klockor} \\       
\melo{Melodi: Det var en lørdag aften}

\songtext{}Jag äter inte klockor, \\
men dricker gärna ur. \\

%-------------------------------------------

\song{Toj hemtegubbar} \\       
\melo{Melodi: Hej, tomtegubbar}

\songtext{}
\leftrepeat Toj hemtegubbar gla i slåsen,\\
och loss tå vastiga lura! \rightrepeat \\
En tiden lid vi heva lär \\
med möcket myda och svärt bestor. \\
Toj hemtegubbar gla i slåsen, \\
och loss tå vastiga lura! \\

%-------------------------------------------

\song{Tussan lull} \\       
\melo{Melodi: Byssan lull}

\songtext{}
\leftrepeat Tussan lull, utav brännvin blir man full, \\
och slipsen man doppar i smöret. \rightrepeat \\
Och näsan den blir röd, \\
och ögonen får glöd, \\
men tusan så bra blir humöret.

\newpage 

%-------------------------------------------

\song{O.P. river} \\       
\melo{Melodi: Ol' man river}

\songtext{}
O.P. river \\
ja, O.P. river \\
var gång jag lenat \\
min hals med Renat \\
jag sagt med ivers \\
att O.P. river \\
långt mer, \\
långt mer. \\
Mången glädes \\
när han fått Sädes \\
och fattighjonet \\
ses le mot Kronet, \\
men faktum bliver \\
att O.P. river \\
långt mer, \\
långt mer. \\

%-------------------------------------------
\song{Magen brummar} \\       
\melo{Melodi: Broder Jakob
}

\songtext{}
Magen brummar, jag försummar\\
hälla dit mera sprit.\\
Nu så ska vi dricka,\\
så att vi får hicka,\\
mera sprit, akvavit.

\newpage 

%-------------------------------------------

\song{Tänk om man hade} \\       
\melo{Melodi: Hej, tomtegubbar}

\songtext{}
\leftrepeat Tänk om man hade lilla nubben \\
uppå ett snöre i halsen. \rightrepeat \\
Man kunde dra den upp och ner \\
så att det kändes som många fler. \\
Tänk om man hade lilla nubben \\
uppå ett snöre i halsen. \\

%-------------------------------------------

\song{För att människan} \\       
\melo{Melodi: Bä bä, vita lamm}

\songtext{}För att människan \\
skall trivas på vår jord \\
bör hon ständigt ha \\
på sitt smörgårsbord: \\
en stor, stor sup åt far, \\
en liten snaps åt mor, \\
och två små droppar \\
åt lille, lille bror. \\

%-------------------------------------------

\song{Fans hämnd} \\       
\melo{Melodi: Där som sädesfälten}

\songtext{När som sädesfälten} böja sig för vinden... .\\
står nån djävul där och böjer dem tillbaks! 
\newpage

%-------------------------------------------

\song{'rännvinskokaren} \\       
\melo{Melodi: En sockerbagare här bor i staden}
\author{Text: Lars T. Johansson, Ehrling Eliasson }

\songtext{}En 'rännvinskokare \\
här bor i skogen. \\
Han säljer 'rännvinet \\
svart till krogen. \\
Å ä' du nykter \\
så kan du gå. \\
Men ä' du fuller \\
så kan du int'. \\

%-------------------------------------------

\song{Jag tror, jag tror} \\       
\melo{Melodi: Jag tror, jag tror på sommaren
}

\songtext{}
Jag tror, jag tror på akvavit, \\
jag tror, jag tror på dynamit, \\\
den ger en kraft att sjunga ut \\
och ingen krämpa blir akut. \\
Man glömmer vardagslivets jäkt \\
och känner stundens ruseffekt. \\
En snaps, en skål, en truddelutt \\
och sen så tar vi våran hutt.

\newpage 

%-------------------------------------------

\song{Perspektiv} \\       
\melo{Melodi: Skånska slott och herresäten}
\author{Text: Quiz och Jacke}

\songtext{}
Ett snapsglas är halvtomt för bittert besvikna \\
Och halvfullt för nöjda, mig kvittar det lika \\
En halvdrucken nubbe är mot min natur \\
Nej, halvfullt ska fyllas, halvtomt drickas ur 

\songinfo{* Var ett av bidragen som tävlade i VM i nyskrivna snapsvisor 2019.} \\



\newpage 

%-------------------------------------------

\song{Gräv ur tundran} \\       
\melo{Melodi: Katjuscha}
\author{Text: Kenneth Hagås}

\songtext{}
Gräv ur tundran två dussin potäter, \\
låt dem jäsa uti fjorton dar. \\
\leftrepeat Modersmjölken för ryssar och sovjeter \\
brännes i babusjkas samovar \rightrepeat \\
Kyl sen drycken i Sibiriens tjäle, \\
tappa upp på immiga små glas. \\
Höj sen glasen för fosterlandets välgång, \\
sjung Nastarovja med en mäktig bas! \\
Höj sen glasen för fosterlandets välgång, \\
sjung Nastarovja – \\
låt glasen gå i kras!

\songinfo{* Den här visan skrevs till ekonomspexet "Lenin eller Gossplanen eller Wanted; Red or alive" 1989. Den blev så uppskattad att den numera ofta sjungs som sista nubbevisa. Första versen och andra fram till ”Höj sen...” sjunges viskande. På ”Höj” höjes även rösten. Meningen är att nubbeglaset kastas över vänster axel efter supen, men det görs av förklarliga skäl aldrig. Nubbeglas i plast brukar däremot krossas i handen på ordet ”kras”.}

\newpage
