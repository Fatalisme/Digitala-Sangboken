
% Headertext
\headertext{vin}

\song{Feta fransyskor}        
\melo{Melodi: Schuberts Marche Militaire}
\tags{FRANKRIKE}

\songtext{
Feta fransyskor som svettas om fötterna  
de trampa druvor som sedan ska jäsas till vin.
Transpirationen viktig e' 
ty den ger fin bouquet.
Vårtor och svampar följer me' 
men vad gör väl de'? 

För... vi vill ha vin, vill ha vin, vill ha mera vin! 
Även om följderna blir att vi må lida pin. 
<b><i>Damerna:</i></b>Flaskan och glaset gått i sin. 
<b><i>Herrarna:</b></i>Hit med vin, mera vin!
<b><i>Damerna:</i></b>Tror ni att vi är fyllesvin?
<b><i>Herrarna:</i></b>JA! (Fast större) }

\songinfo{* Skriven av K-LTH till Sångarstriden 1985. Enligt legenden var de två sista orden enbart riktade till personen som skulle trycka upp sångbladen.} \\

%-------------------------------------------

\song{Gums visa till vinet} 
\author{Text: Carl Sebardt}

 \songtext{\leftrepeat Skål, go vänner i vinet en skål.
Det hjälper oss sorgernan glömma.
För glädjen idag, i vännernas lag,
för detta vi glaset nu tömma. \rightrepeat
Skål, skål, skål, skååål.}

\newpage
%-------------------------------------------

\song{Sudda, sudda}        
\melo{Melodi: Sudda Sudda bort din sura min}

\songtext{
Sudda, sudda, sudda, sudda bort din sura min,  
med fyra jättestora bamseklunkar ädelt vin.   
Munnen den ska sjunga och va' glad
för att den ska bli som den ska va.
Vad häller du då bak det dolda flinet? Vinet! 
Som suddar, suddar bort din sura min.}   



%-------------------------------------------

\song{Bordeaux, bordeaux}       
\melo{Melodi: I sommarens soliga dagar}

\songtext{Jag minns än idag hur min fader, 
kom hem ifrån staden så glader 
och rada' upp flaskor i rader 
och sade nöjd som så: Bordeaux, Bordeaux. 
Han drack ett glas, 
kom i extas, 
och sedan blev det stort kalas. 
Och vi små glin, 
ja, vi drack vin, 
som första klassens fyllesvin. 
Och vi dansade runt där på bordet 
och skrek så vi blev blå: Bordeaux, Bordeaux!}
 
\newpage
%-------------------------------------------

\song{Pussvisa}       
\melo{Melodi: Längtan till landet}

\songtext{Vintern rasat enligt alla källor 
våren kommer när den nu får tid.
Men vi har ju vin och vackra fjällor 
och vi kysser den vi har bredvid. 
Här kan man passa på att pussa någon... 
Snart är vinet där det gör någon nytta 
om du bara fattar glaset i hand. 
Vänd det upp och ned, som rakt i en bytta, 
tänk dig nu att strupen står i brand!}


%-------------------------------------------

\song{Så länge rösten är mild}        
\melo{Melodi: Så länge skutan kan gå}

\songtext{Så länge rösten är mild,
så länge ingen är vild.
Så länge spegeln på väggen
ger halvskaplig bild.
Så länge alla kan gå,
så länge alla kan stå,
så länge alla kan tralla - så fyller vi på.
För vem har sagt att just du kom med storken,
för att bli glad av att lukta på korken?
Men kring vårt bord här nånstans,
vi höjer bägarn med glans,
och låter vinet gå ner i en yrande dans.}

%-------------------------------------------
\song{Imsig vimsig}
\melo{Melodi: Imse vimse spindel}

\songtext{} Imsig, vimsig blir man utav lite vin
klättrar upp på stolen, verkar pigelin.
Ramlar under bordet, sussar en minut.
Vaknar av att vinet i glaset tagit slut.