
% Headertext
\headertext{klassiskt}

\song{Nationalsången} \\
\author{Text: Richard Dybeck}

\songtext{}Du gamla, Du fria, Du fjällhöga nord\\
Du tysta, Du glädjerika sköna!\\
Jag hälsar Dig, vänaste land uppå jord,\\
\leftrepeat Din sol, Din himmel, Dina ängder gröna. \rightrepeat

Du tronar på minnen från fornstora dar,\\
då ärat Ditt namn flög över jorden.\\
Jag vet att Du är och Du blir vad Du va'\\
\leftrepeat Ja, jag vill leva jag vill dö i Norden. \rightrepeat

Jag städs vill Dig tjäna mitt älskade land,\\
Din trohet till döden vill jag svära.\\
Din rätt, skall jag värna, med håg och med hand,\\
\leftrepeat Din fana, högt den bragderika bära. \rightrepeat

Med Gud skall jag kämpa, för hem och för härd,\\
för Sverige, den kära fosterjorden.\\
Jag byter Dig ej, mot allt i en värld.\\
\leftrepeat Nej, jag vill leva jag vill dö i Norden. \rightrepeat

%% Info-text
\songinfo{* Sången är skriven till den västmanländska folkmelodin "Så rider jag mig över tolvmilan skog..." och framfördes första gången vid Dybecks första "aftonunderhållning med nordisk folkmusik" i Stockholm den 18 november 1844. Då började sången "Du gamla, Du friska" men detta ändrades senare. Sången kom med tiden att betraktas som Sveriges nationalsång.}

\newpage
%-------------------------------------------------------------
\song{Studentsången} \\
\melo{Melodi: Prins Gustaf}
\author{Text: Herman Sätherberg}

\songtext{}Sjung om studentens lyckliga dag\\
låtom oss fröjdas i ungdomens vår!\\
Än klappar hjärtat med friska slag,\\
och den ljusnande framtid är vår.\\
Inga stormar än,\\
i våra sinnen bo.\\
Hoppet är vår vän,\\
och vi dess löften tro,\\
när vi knyta förbund i den lund,\\
där de härliga lagrarna gro,\\
där de härliga lagrarna gro!

Svea, vår moder, hugstor och skön,\\
manar till bragd som i forntida dagar.\\
Vinkar med segerns och ärans lön,\\
men den skörd utan strid man ej tar.\\
Aldrig slockne då,\\
känslornas rene brand.\\
Aldrig brista må,\\
vår trohets helga band,\\
så i gyllene frid som i strid:\\
Liv och blod för vårt fädernesland!\\
Liv och blod för vårt fädernesland!

\songinfo{* Marschen komponerades troligen 1851, året innan sångarprinsen dog. Texten skrevs något år senare.}


%-------------------------------------------------------------
\song{Nu grönskar det} \\
\melo{Melodi: Johann Sebastian Bach, ur Bondekantaten}
\author{Text: Evelyn Lindström}

\songtext{} 
Nu grönskar det i dalens famn\\
nu doftar äng och lid.\\
Kom med, kom med på vandringsfärd\\
i vårens glada tid!\\
Var dag är som en gyllne skål\\
till bredden fylld med vin.\\
Så drick, min vän, drick sol och\\
doft, ty dagen, den är din!

Långt bort från stadens gråa hus\\
vi glatt vår kosa styr\\
och följer vägens vita band\\
mot ljusa äventyr.\\
Med öppna ögon låt oss se\\
på livets rikedom,\\
som gror och sjuder överallt\\
där våren går i blom. \\

\newpage
%-------------------------------------------------------------
\song{Den blomstertid nu kommer}\\
\author{Text: Israel Kolmodin}

\songtext{} Den blomstertid nu kommer\\
med lust och fägring stor.\\
Du nalkas ljuva sommar,\\
då gräs och gröda gror.\\
Med blid och livlig värma\\
till allt, som varit dött.\\
Sig solens strålar närma,\\
och allt blir återfött.

De fagra blomsterängar\\
och åkerns ädla säd,\\
de rika örtesängar\\
och lundens gröna träd.\\
De skola oss påminna\\
Guds godhets rikedom,\\
att vi den nåd besinna,\\
som räcker året om.

Man hörer fåglar sjunga\\
med mångahanda ljud,\\
skall icke då vår tunga\\
lovsäga Herren Gud?\\
Min själ, upphöj Guds ära\\
stäm upp din glädjesång\\
till den som vill oss nära\\
och fröjda på en gång. 

\songinfo{* Psalm 199. Skrevs 1694 under Kolmodins tid som biskop i Visby och hette från början "En sommarwisa". } 


%-------------------------------------------------------------
\song{Längtan till landet} \\
\melo{Melodi: Otto Lindblad}
\author{Text: Herman Sätherberg}

\songtext{}Vintern rasat ut bland våra fjällar,\\
drivans blommor smälta ner och dö.\\
Himlen ler i vårens ljusa kvällar,\\
solen kysser liv i skog och sjö.\\
Snart är sommarn här i purpurvågor,\\
guldbelagda, azurskiftande\\
ligga ängarne i dagens lågor\\
och i lunden dansa källorne.

Ja, jag kommer! Hälsen, glada vindar,\\
ut till landet, ut till fåglarne,\\
att jag älskar dem till björk och lindar,\\
sjö och berg, jag vill dem återse.\\
Se dem än som i min barndoms stunder,\\
följa bäckens dans till klarnad sjö,\\
trastens sång i furuskogens lunder,\\
vattenfågelns lek kring fjärd och ö.

\songinfo{* Från "Jägarens hvila. Poetiska bilder från skogen, fältet och sjön", 1838.  Har egentligen ytterligare fyra verser. }

\newpage

%-------------------------------------------------------------

\song{Vårvindar friska} \\
\author{Text: Julia Kristina Nyberg alias Euphrosyne}

\songtext{} Vårvindar friska, leka och viska,\\
lunderna kring, likt älskande par.\\
Strömmarna ila, finna ej vila,\\
förrän i havet störtvågen far.

Klappa mitt hjärta, klaga och hör,\\
vallhornets klang bland klipporna dör.\\
Strömkarlen spelar, sorgerna delar\\
vakan kring berg och dal.

Hjärtat vill brista, ack, när den sista\\
gången jag hörde kärlekens röst.\\
Ögonens låga, avskedets plåga,\\
mun emot mun och klappande bröst.

Fjälldalen stod i grönskande skrud.\\
Trasten slog drill på drill för sin brud.\\
Strömkarlen spelar, sorgerna delar\\
vakan kring berg och dal.

\songinfo{* Ursprunglig titel "Den stackars Anna eller Moll-toner från Norrland“. Originalet har ytterligare fyra verser och skrevs på 1830-talet.}



%-------------------------------------------------------------
\song{Uti vår hage} \\
\melo{Text och melodi av traditionellt gotländskt ursprung}

\songtext{} Uti vår hage där växa blå bär.\\
Kom hjärtans fröjd!\\
Vill du mig något, så träffas vi där.\\
Kom liljor och akvileja, kom rosor och saliveja,\\
kom ljuva krusmynta, kom hjärtans fröjd!

Fagra små blommor där bjuda till dans.\\
Kom hjärtans fröjd!\\
Vill du så binder jag åt dig en krans.\\
Kom liljor och akvileja, kom rosor och saliveja,\\
kom ljuva krusmynta, kom hjärtans fröjd!

Kransen den sätter jag sen i ditt hår.\\
Kom hjärtans fröjd!\\
Solen den dalar men hoppet uppgår.\\
Kom liljor och akvileja, kom rosor och saliveja,\\
kom ljuva krusmynta, kom hjärtans fröjd!
\newpage
Uti vår hage finns blommor och bär.\\
Kom hjärtans fröjd!\\
Men utav alla du kärast mig är.\\
Kom liljor och akvileja, kom rosor och saliveja,\\
kom ljuva krusmynta, kom hjärtans fröjd!

\songinfo{* Växterna som nämns har eventuellt ingått i ett medeltida abortmedel. Sången blev allmänt känd genom Hugo Alfvén vars arrangemang framfördes första gången vid en Orphei Drängar-konsert i Stockholm 1923.} \\

%-------------------------------------------------------------
\song{Sverige} \\
\melo{Melodi: Wilhelm Stenhammar}
\author{Text: Verner von Heidenstam}

\songtext{} Sverige, Sverige, Sverige fosterland,\\
Vår längtans bygd, vårt hem på jorden.\\
Nu spelar källorna, där härar lysts av brand,\\*
och dåd blev saga, men med hand vid hand\\
svär än ditt folk som förr de gamla trohetsorden.

Fall julesnö och susa djupa mo!\\
Brinn österstjärna genom junikvällen!\\
Sverige, moder! Bliv vår strid, vår ro,\\
du land där våra barn en gång få bo\\
och våra fäder sova under kyrkohällen!

\songinfo{* Komponerad 1905.}

\newpage

%-------------------------------------------------------------
\song{O gamla klang- och jubeltid} \\
\melo{Melodi: O alte Burschenherrlichkeit}
\author{Svensk text: August Lindh}

\songtext{} O gamla klang- och jubeltid\\
ditt minne skall förbliva\\
och än åt livets bistra strid,\\
ett rosigt skimmer giva.\\
Snart tystnar allt vårt yra skämt,\\
vår sång blir stum, vårt glam förstämt.\\
O, jerum, jerum, jerum.\\
O, quae mutatio rerum!

Var äro de som kunde allt,\\
blott ej sin ära svika.\\
Som voro män av äkta halt\\
och världens herrar lika?\\
De drogo bort från vin och sång\\
till vardagslivets tråk och tvång.\\
O, jerum...

\textbf{\textit{Filosofer:}} Den ene vetenskap och vett,
in i scholares mänger,
\textbf{\textit{Jurister:}} Den andre i sitt anlets svett,
på paragrafer vränger,
\textbf{\textit{Teologer:}} en plåstrar själen, som är skral,
\textbf{\textit{Medicinare:}} en lappar hop dess trasiga fodral;
O, jerum...

Men hjärtat i en sann student,\\
kan ingen tid förfrysa.\\
Den glädjeeld, som där han tänt,\\
hans hela liv skall lysa.

Det gamla skalet brustit har,\\
men kärnan finnes frisk dock kvar,\\
och vad han än må mista,\\
den skall dock aldrig brista!

Så sluten, bröder, fast vår krets,\\
till glädjens värn och ära!\\
Trots allt vi tryggt och väl tillfreds,\\
vår vänskap trohet svära.\\
Lyft bägarn högt och klinga vän!\\
De gamla gudar leva än,\\
bland skålar och pokaler,\\
bland skålar och pokaler!

\songinfo{* "O, jerum..." är ursprungligen från den tyska visan "Was fang ich armer Teufel an" skriven av Jenaer Blatt 1763. Till denna melodi skrev Eugen Höfling sin nya text 1825, vilken den tidigare uppsalastudenten August Lindh översatte till Västmanlands-Dala nations första sångbok 1921. Angående refrängen: Jerum är en omskrivning för Jesus. Quae mutatio rerum betyder "vilken sakernas förändring", alltså betyder refrängen "O jösses, vilken förändring!" Den latinska diftongen ae, som romarna uttalade "aj", har under medeltiden och framåt vanligen har uttalats som "ä". Sjungen av svenskar bör ordet quae uttalas "kvä".}
