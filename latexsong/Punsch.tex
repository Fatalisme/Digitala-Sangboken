
% Headertext
\headertext{punsch}

\song{Punschen kommer (kall)} \\       
\melo{Melodi: Änkevalsen ur Glada änkan}

\songtext{} 
Punschen kommer,\\ 
punschen kommer,\\ 
ljuv och sval.\\ 
Glasen imma, röster stimma\\ 
i vår sal.\\ 
Skål för glada minnen!\\ 
Skål för varje vår!\\ 
Inga sorger finnas mer\\ 
när punsch vi får.  

\songinfo{* Sjunges under utförande av diverse rörelser fram till dess man fått sin punsch, då slutar man sjunga så att serveringspersonalen lätt kan se vilka gäster som fortfarande väntar på de gula dropparna.} \\

%-------------------------------------------

\song{Punschen kommer (varm)} \\       
\melo{Melodi: Änkevalsen ur Glada änkan}

\songtext{} 
Punschen kommer,\\ 
punschen kommer,\\ 
god och varm.\\ 
Vettet svinner, droppen rinner\\ 
ner i tarm.\\ 
Skål för alla minnen!\\ 
Dem vi snart ej ha.\\ 
Då ett glas med simmig punsch\\ 
vi hunnit ta.

\newpage

%-------------------------------------------

\song{Djungelpunsch} \\       
\melo{Melodi: Var nöjd med allt som livet ger}

\songtext{Jag gillar alla tiders punsch}. \\ 
Punsch till frukost, punsch till lunch, \\ 
en punsch till förrätt, varmrätt och dessert. \\ 
Jag gillar punsch för vet du va', \\ 
rent kaffe gör ju ingen gla'. \\ 
Nej, punsch för fulla muggar vill jag ha. \\ 
Med konjak du lockar. \\ 
Den bästa Renault. \\ 
Förlåt om jag chockar \\ 
och tar punsch ändå. \\ 
Och bjuder du på fin likör \\ 
får du ursäkta om det stör. \\ 
Jag väljer hellre Grönsteds Blå, \\ 
en Cederlunds eller Flaggpunsch å \\ 
- kanske har du ren Platin? \\ 
Jag gillar punsch, \\ 
ger du mig punsch så är jag din. \\ 
För evigt din. \\ 

\newpage

%-------------------------------------------

\song{Lilla punschvisan}
\vspace{0.15cm}
\melochtext{} \\
\songtext{Det var en gång jag tänkte}\\ 
att punschen övergiva,\\ 
men det blir aldrig av\\ 
så länge jag får leva.\\ 
För när jag en gång dör\\ 
så står det på min grav:\\ 
"Här vilar en som\\ 
svenska punschen älskat har".\\ 
Jag gillar, jag gillar punschen,\\ 
jag gillar den som punschen skapat har.\\ 
Jag gillar, jag gillar punschen,\\ 
jag gillar punschen och dess far. 

\songinfo{* Här brukar följa ett varierande antal tilläggsverser.} \\
 
%-------------------------------------------

\song{Jag gillar Sören} \\       
\melo{Melodi: Lilla punschvisan}

\songtext{}
Jag gillar, jag gillar Sören, \\
jag gillar den som Sören skapat har.  \\
Med pallyft och kopiator, \\
han är så jävla, jävla bra. \\

\newpage

%-------------------------------------------

\song{Jag gillar punken} \\       
\melo{Melodi: Lilla punschvisan}

\songtext{}
Jag gillar, jag gillar punken, \\
jag gillar den som punken skapat har. \\
Sex pistols, och Magnus Uggla, \\
de är så jävla, jävla bra. \\



%-------------------------------------------

\song{Visa vid torsdagskväll} \\       
\melo{Melodi: Visa vid midsommartid}

\songtext{Du häller ur flaskan en gyllene tår}\\
av punsch ifrån Cederlunds.\\ 
Du lyfter sen' bägarn' och väl du förstår\\ 
att föra den till din mund.\\ 
Ikväll skall du dricka ditt livselixir\\ 
och känna den ljuva punschen som ett vårbjörkeskir.\\ 
I natt skall du bäras av Razor på bår\\ 
och kallas för fyllehund.\\

%-------------------------------------------

\song{Studiemedelsrondo} \\       
\melo{Melodi: Lossa sand}

\leftrepeat \songtext{Vi dricker punsch}till lunch \\
när vi har fått avin. \\
Vi lunchar hela dagen \\
tills kassan gått i sin.. \rightrepeat

\songinfo{* Sjunges snabbare och snabbare ad nauseam.} \\

%-------------------------------------------


\song{Punschkanon}    
\melochtext{}

\songtext{} 
\textbf{\textit{Herrarna:}} \\
\leftrepeat Punsch, punsch, punsch, punsch,\\
punsch, punsch, alla sorters \rightrepeat

\textbf{\textit{Damerna:}}\\
När supen runnit hädan\\
och maten lagts därpå,\\
och kaffet står på bordet,\\
vad väntar vi då på?\\
\leftrepeat Jo punsch, jo punsch\\
och ännu mera punsch. \rightrepeat\\
Ja, den föll oss i smaken,\\
nu ropar vi gutår,\\
och koppen står där naken\\
och väntar på påtår.\\
\leftrepeat Jo punsch, jo punsch\\
och ännu mera punsch. \rightrepeat\\


\newpage

%-------------------------------------------

\song{Sveriges arraktionalhymn} \\       
\melo{Melodi: Du gamla du fria}
\author{Text: Magnus Hartikainen }

\songtext{Du ädla, du friska, du livselixir,}\\ 
med dig vill jag mina läppar blöta.\\
Den svenskaste drycken på jorden, den förbli.\\
Den gyllene, den underbara söta.\\
Den gyllene, den underbara söta.

Som iskall till kaffet du ställs på vårt bord, \\
och även som varm till torsdagslunchen.\\
Jag halsar dig, lenaste drycken uppå jord.\\
Ja, jag vill leva, jag vill dö av punschen\\
Ja, jag vill leva, jag vill dö av punschen\\


%-------------------------------------------

\song{Punsch, punsch} \\       
\melo{Melodi: Ritsch, ratsch, fillibom}

\songtext{} 
Punsch, punsch fillibom-bom-bom,\\
fillibom-bom-bom, fillibom-bom-bom.\\
Punsch, punsch fillibom-bom-bom,\\
fillibom-bom-bom, fillibom.\\
Vi har ju både Cederlunds och Carlshamns Flagg\\
och Grönstedts blå och lilla Caloric.\\
Det blir för trist med sodavatten,\\
sodavatten, sodavatten.\\
Det blir för trist med sodavatten.\\
Ge mig mera punsch!\\


\newpage 

%-------------------------------------------

\song{FestUs punschvisa} \\       
\melo{Melodi: Tomtarnas julnatt}

\songtext{Punschen, punschen rinner genom strupen},\\
ner i djupen.\\
Blandas, konfronteras där med supen,\\
där med supen.\\
Gula droppar stärker våra kroppar:\\
punsch, punsch, punsch.\\

%-------------------------------------------

\song{Punschens lov} \\       
\melo{Melodi: Rövarna i Kamomilla stad}

\songtext{Ja, punschen är och punschen var}\\ 
och punschen skall förbliva\\ 
en lidelse vi alla har\\ 
som inget kan fördriva.\\ 
Ja, punschen tinar upp såväl\\ 
och svalkar både kropp och själ.\\ 
Den botar begären\\ 
och lindrar besvären,\\ 
ja, punschen den gör både gott och väl. 

\songinfo{* Från Kårspexet "Sven Hedin eller en enkel tur och retur" 1987.}

\newpage
%-------------------------------------------

\song{P.U.S.S. Punschvisa} \\       
\melo{Melodi: Öppna din dörr}

\songtext{Den första gång jag prova dig}\\
och kände hur du kom i mig,\\
då föll allting jag trott var sant,\\
nästa gång var det likadant.\\
Du var så sliskig då,\\
som nåt jag aldrig provat på,\\
och hjärtat slår ett slag för var förlorad dag.

Jag gömde mig i fantasi,\\
långt bort från vardans tråkeri,\\
och var gång som jag hade dig,\\
så väcktes skönsång uti mig.\\
Nu står jag här idag,\\
med allt jag nånsin velat ha\\
och hjärtat slår ett slag för var förlorad dag.

Öppna din kork\\
och låt mig sedan dricka dig.\\
Snälla du säg inte nej,\\
du vet att jag vill.\\
Och jag ska öppna din kork,\\
med punsch blir var dag till kalas.\\
Jag ska fylla upp mitt glas\\
och jag ska sjunga till.\\
Så öppna din kork!

\newpage
%-------------------------------------------

\song{Bered din kropp för punschen} \\       
\melo{Melodi: Bered en väg för herren}

\songtext{}
Bered din kropp för frälsning,\\
av guldgul karaktär.\\
Från gudarna en hälsning,\\
till alla männ'skor här.\\
Den bringar fred på jorden,\\
vi stämmer upp i orden.\\
Må gudars vilja ske,\\
men först om punsch vi be.

Se himlens änglar gråter,\\
när punschen tagit slut.\\
De tänka livet hårt är,\\
om ingen köper ut.\\
Men Gud går på systemet,\\
och löser det problemet.\\
Han måste aldrig visa leg,\\
han har mustasch och skägg.

\songinfo{* Ur ekonomspexet "Barabbas", 2002}\\

\newpage
%-------------------------------------------

\song{Änglapunsch} \\       
\melo{Melodi: Änglamark}

\songtext{Kalla den gudagåva}eller himlanektar, vad du vill.\\ 
Punschen den gyllne, de gamle oss skänkte.\\ 
Vet att så länge som punschen nå'nsin funnits till\\ 
glädjen den höjde och sorgerna dränkte.

Blunda och dröm om en blommande sommarnatt\\ 
svala bersåer där punschen står immig.\\ 
Eller en höstdag när Nordan har lekt tafatt,\\ 
varm punsch som ångar och ärtsoppa simmig.

Punschen den älskas ju av alla och envar.\\ 
Låt festen börja - låt punschen få flöda!\\ 
Skål alla vänner som har nå't i glaset kvar,\\ 
hedra nu minnet av gamle kung Oscars da'r!

Kalla den gudagåva eller himlanektar, vad du vill.\\ 
Punschen den gyllne, som får oss att drömma.\\ 
Fukta din strupe, låt inte flaskan få stå still,\\ 
skåla för punschen och glasen vi tömma!\\ 

\newpage

%-------------------------------------------

\song{Visa vid doft av punsch} \\       
\melo{Melodi: Visa vid vindens ängar}
\author{Text: André Mabande}

\songtext{Det går en susning igenom salen;}\\ 
nån' kände doft av en arrakspunsch.\\ 
Och så tystas så alla talen\\ 
när samfällt följande visa sjungs:

Tänk att få smutta några droppar\\ 
av den där bägarens innehåll.\\ 
När vi mot gomseglet drycken måttar\\ 
spelar sorgerna ingen roll.

Det går en susning igenom salen\\ 
när så den ändande strofen sjung'ts.\\ 
Och alla lyfter vi punschpokalen\\ 
och säger skål uti arrakspunsch!\\

%-------------------------------------------

\song{Kaffe, kaffe, kaffe} \\       
\melo{Melodi: Du ska få min gamla cykel när jag dör}

\songtext{Vi har ätit och vi mår så väldans bra}\\  
och nu vill nog alla säkert kaffe ha.\\  
Snart så får vi höra stönen\\  
när vi sjunger kaffebönen.\\
Det ska höras ända bort till nästa sta'.

Kaffe, kaffe, kaffe, konjak och likör\\  
ger åt alla här ett mycket glatt humör.\\  
Och det kan ni ge er katten\\  
vi ska sitta hela natten\\  
dricka kaffe, kaffe, konjak och likör.

Ofta får man höra ordet kaffetant\\  
husets herre säger gärna helt galant\\  
Du min rara, du min sköna,\\  
älskar du din kaffeböna\\  
mer än mig, det kan väl inte vara sant?\\  
Kaffe, kaffe, kaffe...

Calle Schewen blanda kaffet sitt med Kron.\\  
Det var medicin, han hade denna tron.\\  
Och man blir ju allt en rask en\\  
när man dricker kaffekasken.\\  
Jublar högt i skyn och sedan tar man ton.\\  
Kaffe, kaffe, kaffe...\\  

Uti alla väder smakar fikan gott\\  
och hos damer tungan ofta får så brått\\  
och man skulle nog bli häpen\\  
om man kom på kafferepen\\  
munnen går som om den vore smord med flott.\\  
Kaffe, kaffe, kaffe...\\  

\newpage

%-------------------------------------------

\song{Punschvisa} \\       
\melo{Melodi: Med en enkel tulipan}

\songtext{Nu med en ny och stadig krök} \\
med armen gör vi försök \\
att lyfta koppen, att lyfta koppen \\
som står och väntar. \\
Håll blicken fäst vid koppens rand \\
och darra inte på hand. \\
Nu allesammans, nu allesammans\\
på munnen gläntar.\\
En liten punschtår så här placerad\\
i ena handen sig bättre gör\\
än tio liter uppå Systemet\\
och inga pengar att köpa för.\\
Spill inga droppar på ditt bord\\
och spill ej mer några ord.\\
Nu tar vi punschen, nu tar vi punschen\\
som står och väntar.\\

\newpage

%-------------------------------------------

\song{Vädjan till punschen} \\       
\melo{Melodi: Sov du lilla videung}

\songtext{Kom nu lilla punschen min},\\     
följ nu efter supen.\\     
Snart så skall du åka in\\     
ner igenom strupen,\\     
till mitt stora magpalats,\\     
där det finns så mycket plats.\\     
Kom nu lilla punschen,\\     
följ nu efter supen.\\

%-------------------------------------------

\song{När kaffet är serverat} \\       
\melo{Melodi: Mössens julafton}

\songtext{}
När kaffet är serverat och maten tagit slut,\\
och alla dom som blivit allt för fulla kastats ut.\\Då vill vi ha ett nytt glas med något ljust och kallt,\\
som höjer och förbättrar vår promillehalt.

Arrak, etanol och sakaros,\\
med salt och vatten blir den bästa blandning som kan fås.\\
Söt och sliskig, ja, rent utav viskös,\\
en sexton, sjutton glas så blir man medevetslös!\\

\newpage

%-------------------------------------------

\song{Ärter och punsch} \\       
\melo{Melodi: Fritiof och Carmencita}

\songtext{} 
Ärter och punsch, en liten rätt med traditioner.\\
Den smakar bra och väcker många sensationer.\\
Blekgul till färgen, smaken går in till märgen,\\
med en senap som blandas enligt gammal tradition,\\
så ska den njutas denna svenska folks passion,\\
en torsdagskväll i varje månad.

Skål nu vänner uti denna mäss,\\
ärter och punsch ska vi njuta utan stress.\\
Inga sura miner vill vi se i afton,\\
när doften utav ärter och punsch sprids i salongen.\\
Tänk på Grönstedt, Cederlund och Flagg,\\
åh, vilken doft från denna arraksgula tagg.\\
Hela natten ska vi njuta denna underbara brygd,\\
skål för lilla ärten och punschen.\\

\newpage

%-------------------------------------------

\song{Det finns på fester tider} \\       
\melo{Melodi: Min soldat}
\author{Text: Chrischan Johanson}

\songtext{}   
Det finns på fester tider då när allt har sitt slut.\\
Så drick och härja, festa, raggla, sörj inte, njut.\\
För det kommer värre tider, tro mig min vän.\\
Det är korkat att vänta till sen.

Ty vinet är uppdrucket och fördrinken slut\\
och ölet är avslaget och snapsen slogs ut,\\
men det gör det samma för vi har våran punsch...\\
... någonstans i magen!\\
%-------------------------------------------

\song{Johansson är ful} \\       
\melo{Melodi: Kostervals}

\songtext{}Johansson är ful,\\     
han får gömma sig i ett skjul. \\
Ful är Perssons bror; \\
Persson själv har glömt bort vart han bor.

Fulhet gör mig trött, \\
ge mig snabbt någonting som är sött! \\
Ge mig punsch, \\
ett glas med punsch, \\
nej, en flaska med punsch! \\
(Å' en kasse bärs!) \\

\newpage 

%-------------------------------------------

\song{Låtom oss hylla punschen} \\       
\melo{Melodi: Trink, trink, brüderlein trink}

\songtext{Punsch, punsch vi vill ha punsch}
det är en underbar dryck.\\
Punsch, punsch massor av punsch \\
det är vår senaste nyck.\\
Ta fram en flaska och kyl den med is \\
ty annars kan det bli kri-is. \\
Ställ upp i kampen kring halvnykterhet \\
punsch är det bästa vi vet. \\

%-------------------------------------------

\song{Sista punschvisan} \\       
\melo{Melodi: Auld lang syne}

\songtext{När punschen småningom är slut}\\
och vår flaska blivit tom,\\
då vänder vi den upp och ner\\
till dess inget rinner ur.\\
\leftrepeat Så slickar vi, så slickar vi\\
båd' utanpå och i,\\
och finns det ändå något kvar\\
får det va' till sämre dar. \rightrepeat

